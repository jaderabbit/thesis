%%%%%%%%%%%%%%%%%%%%%%%%%%%%%%%%%%%%%%%%%%%%%%%%%
%%%%%%%%%%%%%%%%%%%%%%%%%%%%%%%%%%%%%%%%%%%%%%%%%

\chapter{Symbols}
\label{app:symbols}

The following appendix defines the symbols used in each chapter.  


%%%%%%%%%%%%%%%%%%%%%%%%%%%%%%%%%%%%%%%%%%%%%%%%%
%%%%%%%%%%%%%%%%%%%%%%%%%%%%%%%%%%%%%%%%%%%%%%%%%

\section{Chapter~\ref{chap:second}: Foraging}
\label{sec:symbols:foraging}


\begin{description}
	\item[\parbox{\namewidth}{$p1$}] The probability that a robot will changed from rest to searching is adapted upon return to the sink and from deposit to rest.


\section{Chapter~\ref{chap:divisionoflabour}: Division Of Labour}
\label{sec:symbols:divisionoflabour}


\begin{description}
	\item[\parbox{\namewidth}{$R$}] The set of tasks that can be performed by individuals of the swarm.
	\item[\parbox{\namewidth}{$\gamma$}] An individual robot in a swarm.
	\item[\parbox{\namewidth}{$\nu$}] A task such that $\nu \in R$
	\item[\parbox{\namewidth}{$r_{\gamma,\nu}$}] A response threshold for of robot $\gamma$ for task $\u$
	\item[\parbox{\namewidth}{$z$}] The probability of a robot to switch from search behaviour to rest behaviour.
\end{description}


%%%%%%%%%%%%%%%%%%%%%%%%%%%%%%%%%%%%%%%%%%%%%%%%%
%%%%%%%%%%%%%%%%%%%%%%%%%%%%%%%%%%%%%%%%%%%%%%%%%

\section{Chapter~\ref{chap:third}: Nature Inspired Algorithms for Prioritized Foraging}
\label{sec:symbols:foraging}
node[anchor=west] {$_ \geq $} (dance);
	\draw[arrow] (dance) to[bend left=20] node[anchor=east] {$\varrho \geq \rho$} (forage);
	\draw [arrow] (dance) -- node[anchor=south] {$\varrho < \rho$} (scout)

\begin{description}
\setlength{\itemsep}{-1mm}
	\item[\parbox{\namewidth}{$i$}] The current time step in a swarm robot experiment
	
\item[\parbox{\namewidth}{$state$}] The state a particular robot is in.

\item[\parbox{\namewidth}{$role$}] The role of a particular robot in the honey bee algorithm.

	\item[\parbox{\namewidth}{$i_{state}$}] The current time step of being in consecutive state $state$, where $state$ is any valid state for the robot.
	
	\item[\parbox{\namewidth}{$t_{wait}$}] The maximum time a robot can spend consecutively in wait state.
	
	\item[\parbox{\namewidth}{$t_{ls}$}] The maximum time a robot can spend consecutively in the local search state.
	
	\item[\parbox{\namewidth}{$t_{forage}$}] The maximum time a robot can spend consecutively in the forage state.

	\item[\parbox{\namewidth}{$t_{explore}$}] The maximum time a robot can spend consecutively in the explore state


	\item[\parbox{\namewidth}{$t_{dance}$}] The maximum time a robot can spend consecutively in the recruitment state.
	
	
	\item[\parbox{\namewidth}{$\varsigma$}] The type of an item which can either be prioritized or non-prioritized.

	\item[\parbox{\namewidth}{$\mu_\varsigma$}] The evaluation of the quality of a site of type $\varsigma$.
	
	\item[\parbox{\namewidth}{$\Phi$}] site quality threshold determining whether a the scout robot switches into the dance state. 
	
	\item[\parbox{\namewidth}{$v$}] A robot's path integration vector

	\item[\parbox{\namewidth}{$\omega$}] A memorized path integration vector representing the location of a site.
	
	\item[\parbox{\namewidth}{$\vartheta$}] An item found by a robot.
	\item[\parbox{\namewidth}{$\xi$}] The site where an item is found.

	\item[\parbox{\namewidth}{$\varrho$}] A random number selected from a uniform distribution such that $\varrho\in(0,1)$.
	
	\item[\parbox{\namewidth}{$\rho$}] The probability that a scout robot becomes a forager robot.
	
\item[\parbox{\namewidth}{$\k_j$}] is the value captured on the $j-$th distance sensor of a robot.

\item[\parbox{\namewidth}{$n$}] The number of distance sensors.

\item[\parbox{\namewidth}{$d$}] The direction a robot must take to get to the sink.

\item[\parbox{\namewidth}{$\alpha$}] The probability of listening to the details communicated by the scout robot.		

\item[\parbox{\namewidth}{$f_{max}$}] The maximum time a robot will forage for prioritized items, without finding any before switching to foraging a non-prioritized type.

\item[\parbox{\namewidth}{$X$}] The percentage of robots initialized as scout robots.
\end{description}

%%%%%%%%%%%%%%%%%%%%%%%%%%%%%%%%%%%%%%%%%%%%%%%%%
%%%%%%%%%%%%%%%%%%%%%%%%%%%%%%%%%%%%%%%%%%%%%%%%%

\section{Chapter~\ref{chap:experiment}: Experimental Setup}
\label{sec:symbols:foraging}


\begin{description}
	\item[\parbox{\namewidth}{$d$}] Desirability of a direction $i$.
	\item[\parbox{\namewidth}{$i$}] A direction a robot can perceive.
	
	\item[\parbox{\namewidth}{$\kappa_i$}] The clarity which indicates the distance of next nearest obstacle.
	 
	\item[\parbox{\namewidth}{$v$}] The depth of view of a robot's perception.

	\item[\parbox{\namewidth}{$\iota_i$}] The directness of a direction $i$, calculated as the angular deviation from the direction of the destination

	\item[\parbox{\namewidth}{$f$}] The	field of view of a robot's perception.
		
	\item[\parbox{\namewidth}{$\lambda$}] A ratio which determines whether clarity, $\kappa_i$ or directness, $\iota_i$ of direction $i$, has more effect on desirability $d$.
	
	\item[\parbox{\namewidth}{$S$}] The	size of the environment grid.

	\item[\parbox{\namewidth}{$p$}] The density of the items on the grid.

	\item[\parbox{\namewidth}{$r$}] The ratio of prioritized to non-prioritized items.

	\item[\parbox{\namewidth}{$\tau$}] The ratio of robots foraging prioritized items to the ratio of robots foraging non-prioritized items.

	\item[\parbox{\namewidth}{$c$}] The density of robots.

	\item[\parbox{\namewidth}{$\sigma$}] The percentage of prioritized items foraged over time.

	\item[\parbox{\namewidth}{$\mu$}] The percentage of non-prioritized items foraged over time.

	\item[\parbox{\namewidth}{$\epsilon$}] The average time spent by agents waiting at the sink.
	
\end{description}

The effect of the local attractor is modelled by  evaluating each direction in the field of view, to select the most desirable direction. Desirability, $d$, of a direction, $i$,