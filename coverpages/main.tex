%%%%%%%%%%%%%%%%%%%%%%%%%%%%%%%%%%%%%%%%%%%%%%%%%
%%%%%%%%%%%%%%%%%%%%%%%%%%%%%%%%%%%%%%%%%%%%%%%%%

\newlength{\negativetitlepageoffset}
\setlength{\negativetitlepageoffset}{-5cm}

\begin{titlepage}
	\ \vspace{\negativetitlepageoffset}
	\vspace{\stretch{1}}
	\setlength{\baselineskip}{2.4\originalbaselineskip}
	\begin{center}
		\textsf{\huge Nature Inspired Algorithms for Prioritized Foraging}
	\end{center}
	\begin{center}
		\textsf{by}
	\end{center}
	\begin{center}
		\textsf{\large Jade Zo\"e Abbott}
	\end{center}
	\vspace{\stretch{1}}
	\setlength{\baselineskip}{1.3\originalbaselineskip}
	\begin{center}
		\textsf{Submitted in partial fulfilment of the requirements for the degree\\
		Master of Science (Computer Science)\\
		in the Faculty of Engineering, Built Environment and Information Technology\\
		University of Pretoria, Pretoria}
	\end{center}
	\vspace{0.15cm}
	\centerline{
		\textsf{November 2017}
	}
\end{titlepage}

%%%%%%%%%%%%%%%%%%%%%%%%%%%%%%%%%%%%%%%%%%%%%%%%%
%%%%%%%%%%%%%%%%%%%%%%%%%%%%%%%%%%%%%%%%%%%%%%%%%

\pagestyle{empty}
\newpage

\textsf{\small
	\vfill
	\noindent Publication data:\\*[2.5mm]
	\parbox{\textwidth}{
		\fontsize{9}{10pt}
		\selectfont
		Jade Zo\"e Abbott. Nature Inspired Algorithms for Prioritized Foraging. Masters thesis, University of Pretoria, Department of Computer Science, Pretoria, South Africa ,December 2013.
	}\\*[10.5mm]
	Electronic, hyperlinked versions of this thesis are available online, as Adobe PDF files, at:\\*[2.5mm]
	\parbox{\textwidth}{
		\fontsize{9}{9.5pt}
		\selectfont
		\url{http://cirg.cs.up.ac.za/}\\*[1.5mm]
		\url{http://upetd.up.ac.za/UPeTD.htm}
	}
}

%%%%%%%%%%%%%%%%%%%%%%%%%%%%%%%%%%%%%%%%%%%%%%%%%
%%%%%%%%%%%%%%%%%%%%%%%%%%%%%%%%%%%%%%%%%%%%%%%%%

\newpage

\begin{center}
	{\large\bf Nature Inspired Algorithms for Prioritized Foraging}
\end{center}
\begin{center}by\end{center}
\begin{center}
	{Jade Zoe Abbott}\\
	\ifpdf
		E-mail: \href{mailto:jabbott@cs.up.ac.za}{jabbott@cs.up.ac.za}
	\else
		E-mail: jabbott@cs.up.ac.za
	\fi
\end{center}
\vspace{1cm}
\begin{center}{\large\bf Abstract}\end{center}


Foraging is a major problem in swarm robotics, which has been applied to many areas such as agriculture, and search and rescue. This dissertation defines a variation of swarm robotics foraging, called prioritized foraging. Prioritized foraging differs from other foraging problems, in that there are two types of items: prioritized items and non-prioritized items. Furthermore, a novel honey bee inspired foraging algorithm is developed. An empirical analysis of three foraging algorithms (a na\"ive algorithm, and two nature inspired algorithms, namely a desert ant inspired algorithm and the novel honey bee inspired algorithm) is performed on the prioritized foraging problem. The analysis investigates each algorithm's performance on the prioritized foraging problem in terms of the major swarm robotics characteristics of efficiency, scalability, flexibility, and robustness as well as the behaviours that enable those characteristics. The work concludes that the honey bee algorithm is highly efficient, highly flexible, highly scalable in terms of problem density, and most robust in terms of redundancy due to the algorithm's division of labour between prioritized and non-prioritized items. The desert ant algorithm is almost as efficient as the honey bee algorithm. The na\"ive algorithm has very poor efficiency, compared to the other algorithms, but was the most scalable in terms of swarm size. Both the desert ant and na\"ive algorithm experience poor flexibility, scalability in terms of problem density, and robustness in terms of redundancy.

\noindent\

\noindent{\bf Keywords:} Swarm Robotics, Foraging, Prioritized Foraging, Desert Ant, Honey bee

\vfill
\noindent
{\bf\parbox{26.8mm}{Supervisor}:} Prof. A. P. Engelbrecht \\* % 
{\bf\parbox{26.8mm}{Department}:} Department of Computer Science \\*
{\bf\parbox{26.8mm}{Degree}:} Master of Science

%%%%%%%%%%%%%%%%%%%%%%%%%%%%%%%%%%%%%%%%%%%%%%%%%
%%%%%%%%%%%%%%%%%%%%%%%%%%%%%%%%%%%%%%%%%%%%%%%%%

\newpage

\ \vspace{\stretch{1}}

\begin{quotation}
``It may be that. You never can tell with bees.\ldots''
\end{quotation}
\begin{flushright}
Winnie the Pooh by A. A. Milne
\end{flushright}

\vspace{1cm}

\begin{quotation}
``The Three Laws of Robotics:

1: A robot may not injure a human being or, through inaction, allow a human being to come to harm;

2: A robot must obey the orders given it by human beings except where such orders would conflict with the First Law;

3: A robot must protect its own existence as long as such protection does not conflict with the First or Second Law;''
\end{quotation}
\begin{flushright}
Isaac Asimov, I, Robot
\end{flushright}


\ \vspace{\stretch{1}}

%%%%%%%%%%%%%%%%%%%%%%%%%%%%%%%%%%%%%%%%%%%%%%%%%
%%%%%%%%%%%%%%%%%%%%%%%%%%%%%%%%%%%%%%%%%%%%%%%%%

\newpage

\begin{center}{\Large\bf Acknowledgements}\end{center}

\vspace{0.3cm}

\noindent I wish to express the greatest thanks to the following individuals and organisations for their support throughout my research:
\begin{itemize}
	\item Prof Andries P. Engelbrecht for the amazing guidance and never giving up on me, despite my total loss of motivation, as well as the financial support.
	\item To my mother and father for their constant love, belief and encouragement.
	\item Bernard Frankel for sitting next to me and bringing me things, while I clawed my eyes out every Saturday and Sunday.
	\item Theo Crous for reminding me that the problem is not as big as my mind made it out to be and forcing me to get on with it.
	\item M-club (particularly Kristina Young, Christien Kroeze and Christopher Cleghorn) for the problem solving over wine and tears.
	\item Retro Rabbit for the financial support and giving me confidence in my abilities as a computer scientist.
	\item Wolves Cafe for always having good coffee, wifi, excellent seating, power and allowing me to sit there all day. It made research bearable.
	\item The National Research Foundation for the financial support. Opinions expressed in this thesis and arrived at, are those of the author and not necessarily the National Research Foundation.
\end{itemize}

%%%%%%%%%%%%%%%%%%%%%%%%%%%%%%%%%%%%%%%%%%%%%%%%%
%%%%%%%%%%%%%%%%%%%%%%%%%%%%%%%%%%%%%%%%%%%%%%%%%

\cleardoublepage
\pagestyle{plain}
\pagenumbering{roman}
\setcounter{page}{1}
\ifpdf
\pdfbookmark[0]{Contents}{contents}
\fi
\tableofcontents

%%%%%%%%%%%%%%%%%%%%%%%%%%%%%%%%%%%%%%%%%%%%%%%%%
%%%%%%%%%%%%%%%%%%%%%%%%%%%%%%%%%%%%%%%%%%%%%%%%%

\cleardoublepage
\ifpdf
\phantomsection
\fi
\addcontentsline{toc}{chapter}{List of Figures}
\listoffigures

%%%%%%%%%%%%%%%%%%%%%%%%%%%%%%%%%%%%%%%%%%%%%%%%%
%%%%%%%%%%%%%%%%%%%%%%%%%%%%%%%%%%%%%%%%%%%%%%%%%


%%%%%%%%%%%%%%%%%%%%%%%%%%%%%%%%%%%%%%%%%%%%%%%%%
%%%%%%%%%%%%%%%%%%%%%%%%%%%%%%%%%%%%%%%%%%%%%%%%%

\cleardoublepage
\ifpdf
\phantomsection
\fi
\addcontentsline{toc}{chapter}{List of Algorithms}
\listof{algorithm}{List of Algorithms}

%%%%%%%%%%%%%%%%%%%%%%%%%%%%%%%%%%%%%%%%%%%%%%%%%
%%%%%%%%%%%%%%%%%%%%%%%%%%%%%%%%%%%%%%%%%%%%%%%%%

\cleardoublepage
\ifpdf
\phantomsection
\fi
\addcontentsline{toc}{chapter}{List of Tables}
\listoftables

%%%%%%%%%%%%%%%%%%%%%%%%%%%%%%%%%%%%%%%%%%%%%%%%%
%%%%%%%%%%%%%%%%%%%%%%%%%%%%%%%%%%%%%%%%%%%%%%%%%