%%%%%%%%%%%%%%%%%%%%%%%%%%%%%%%%%%%%%%%%%%%%%%%%%
%%%%%%%%%%%%%%%%%%%%%%%%%%%%%%%%%%%%%%%%%%%%%%%%%

\chapter{Foraging}
\label{chap:second}

%%%%%%%%%%%%%%%%%%%%%%%%%%%%%%%%%%%%%%%%%%%%%%%%%
%%%%%%%%%%%%%%%%%%%%%%%%%%%%%%%%%%%%%%%%%%%%%%%%%

In swarm robotics, robot foraging has become a benchmark problem due to its complex nature involving the coordination of numerous sub-tasks. Robot foraging is also one of the swarm robotics problems that has some very obvious useful applications. This chapter aims to provide an in-depth review of what foraging is, how different social insects forage as well as to provide a review of existing foraging algorithms in swarm robotics. A definition of foraging in swarm robotics is provided in Section~\ref{sec:second:definition}, while Section~\ref{foraginginnature} describes foraging behaviours observed in nature, specifically in bees and ants. Section~\ref{foraging:foraginginswarmrobotics} presents a taxonomy for swarm robot foraging and discusses existing swarm robotics foraging algorithms. Prioritized foraging is defined and discussed in Section~
\ref{prioritizedforaging}, and the chapter is summarized in Section~\ref{foraging:summary}.


%%%%%%%%%%%%%%%%%%%%%%%%%%%%%%%%%%%%%%%%%%%%%%%%%
%%%%%%%%%%%%%%%%%%%%%%%%%%%%%%%%%%%%%%%%%%%%%%%%%

\section{Definition of Foraging}
\label{sec:second:definition}

Foraging was initially studied by biologists, particularly the foraging behaviour of ants \cite{holldobler1990ants,bernstein1974seasonal}, bees \cite{seeley2009wisdom} and bacteria \cite{resnick1994turtles}. Foraging has a variety of real-life applications such as search and rescue \cite{jennings1997cooperative,murphy2000biomimetic} and waste clean-up \cite{balch1995io}. Numerous swarm robotics algorithms have been developed to improve robustness, time, and energy efficiency of the foraging process for multiple variations of the foraging problem as outlined in \cite{winfield2009foraging}. 

Foraging is the act of searching for and collection (or capturing) of food for storage and consumption \cite{winfield2009foraging}. Foraging is an important problem that was first addressed by biologists in the examination of nature, particularly with the foraging behaviour of ants and bacteria. In a robot context, foraging is defined as the search and collection of scattered objects in an environment and returning those objects to a collection point.

Foraging sub-tasks include the efficient search and collection of food, homing whilst carrying food to the nest site, and then depositing of the food item in the nest before returning to the foraging task. Efficient foraging requires indirect or direct co-operation between individuals in order to transport food items too large for individual transport. The foraging process has been adapted for numerous foraging problems and for different types of robots.


\section{Foraging in Nature}
\label{foraginginnature}
As with many swarm robotics problems, the inspiration for foraging comes from nature, in particular, social insects such as ants and honey bees. This section describes the biological inspiration for foraging. Section~\ref{biological:ants} describes foraging in ants, while foraging activities of bees are discussed in Section~\ref{bees:biologicalinspiration}. Section~\ref{foraging:other} discusses other types of foraging techniques seen in nature.


\subsection{Ant Foraging Behaviour}
\label{biological:ants}
% Intro 
As with many swarm robotics problems, the inspiration for foraging comes from nature, in particular, social insects such as ants and honey bees. This section describes the biological models that are used in the algorithms derived by this thesis. 

%Ants
The study of ants has revealed that impressive emergent activities can be achieved with simple interacting agents with only a few simple rules. Many ant species use a form of communication known as stigmergy \cite{dorigo2000ant}. Ants perform indirect communication between ants by depositing a substance known as pheromone. In foraging, pheromone is deposited on the paths between the nest and the food source. Other ants detect the deposited pheromone and will prefer to follow paths which have a greater amount of pheromone deposit. Ant pheromones have been modelled in numerous swarm intelligence algorithms such as ant systems \cite{dorigo2006ant, dorigo2010ant}. 
 
Although algorithms based on ant foraging behaviour are used to solve optimization problems, the algorithms are often difficult to replicate in a real-life robot environment, since most of the algorithms need to model pheromone dropping and detection. A robotics algorithm that uses pheromone requires robots to be equipped with a substance-distributors, beacon-deployers or complex communication that simulates pheromone deposition and trail-following \cite{hoff2010two}.

The desert seed-harvester ant (\textit{Pogonomyrmex ant}) does not make use of stigmergy to forage, because pheromone deposited on desert sand would be blown away by the wind \cite{collett1992visual,hecker2015beyond}. Instead, desert ants use a technique called path integration for navigation to relocate food sources that have already been found by random exploration \cite{collett1998local,wehner2003desert}. Fig~\ref{pathintegration} demonstrates path integration. The black solid lines represent the path of the ant and the blue dotted lines show how the direction to the nest is maintained as the ant explores. The ant is constantly monitoring the change in heading from the original heading such that, at the destination, the ant has a direction directly back to the sink. As a result, a shortcut back to the nest is calculated in order to minimize the heat stress caused by walking through the hot desert. The shortcut back to the nest is known as the home vector \cite{muller1988path}. Path integration, more commonly known as dead reckoning, is a key evolutionary aspect gained from living in the hot barren desert. When path integration fails, the ant resorts to using landmarks, such as the sun, for navigation \cite{collett1998local}.

\begin{figure} [h]
	\centering
	\includegraphics[width=\textwidth]{chapters/chapter2/figures/drawing.png}
	\caption{Path Integration}
	\label{pathintegration}
\end{figure}

Due to the lack of pheromone, the desert seed-harvester ant was simpler to model for real-robot interaction. Desert seed-harvester ant foraging has been modelled in \cite{moller1998modeling,hecker2012formica}. Hecker \textit{et al} \cite{hecker2012formica} performed experiments to compare the performance of two foraging algorithms: The one algorithm was based on desert ant foraging which has no pheromone-like communication and the other algorithm included pheromone-like communication. It was shown that communication improved performance; however, the desert ant based algorithm still performed comparatively well. 

\subsection{Bee Foraging Behaviour}
\label{bees:biologicalinspiration}
Bees have an impressive set of abilities considering the simplicity of a single individual. They have the ability to remember the colour and shape of flowers \cite{zhang2006honeybee}, and in terms of navigation, they are able to learn local features and routes due to well-developed learning and memorizing capacities \cite{menzel2001cognitive}. Honey-bees are even time aware \cite{moore1989influence}. 

Honey bees have efficient division of labour between different functions in the hive, such as foraging and brood-care. Within the foraging activity itself, honey bees perform foraging-specific division of labours. Honey bee foraging is made up of three different roles namely employed foraging, unemployed foraging and scouting \cite{seeley2009wisdom}. Scouts explore the environment to locate new food sources. Once a source has been found, scouts return to the hive to communicate information about the located food source. To communicate the foraging information, scout bees perform a waggle-dance at the hive. The waggle-dance communicates information about the distance and bearing of the resource. Recruitment \cite{seeley2009wisdom} refers to the process of communicating high quality foraging sites to the swarm.

Unemployed foragers wait on the dance floor, and evaluate the dances of the scout bees. An unemployed bee selects a location described by the scout bees, and become an employed forager. Employed forager bees use the information about the resource, attempt to locate the source, load themselves with food, and return to the hive where unemployed foragers are ready to offload the food. Jansen \cite{janson2007searching} suggests that unemployed bees become exploring scouts when they do not detect any dancing scout bees. 


\subsection{Other Foraging Behaviours}
\label{foraging:other}

This thesis focuses on ant and bee foraging behaviours. Other foraging behaviours exist, some of which are briefly discussed in this section.

The $E. coli$ bacteria forage for carbohydrates in the human gut. Bacteria have a unique search mechanism whereby they move in the same direction for a period of time (which is called a ``run") and then change direction using their complex flagella (known as a ``tumble"). Bacteria exhibit the behaviour of continuing to move in a direction if the nutrient gradient is positive in that direction. Bacteria stop moving in a direction if the nutrient gradient of that direction is negative in order to avoid toxic substances. Under certain conditions, the bacteria secrete an attractant substance which attracts the bacteria to one another for protection. The foraging behaviour of $E. coli$ bacteria inspired a computational intelligence algorithm \cite{passino2010bacterial}.

In nature, a predator animal persues a prey animal in order to forage. Predator-prey foraging behaviour has been extensively studied in the field of optimal foraging theory, which aims to predict the behaviour of an animal searching for food \cite{charnov1976optimal}. Predator-prey models have been used in other swarm intelligence techniques in multi-objective optimization problems \cite{nolfi1998coevolving}.

Particle swarm optimization is loosely inspired by the flocking behaviour of birds when foraging for food \cite{kennedy1995particle}.

\section{Foraging in Swarm Robotics}
\label{foraging:foraginginswarmrobotics}
This section addresses foraging in a swarm robotics context. Firstly, a taxonomy of robot foraging is presented in Section~\ref{sec:second:taxonomy} and then existing nature-inspired swarm robotics foraging algorithms are discussed in Section~\ref{sec:second:natureinspiredalgs}. The chapter concludes with a discussion of the challenges faced by swarm robotics in Section~\ref{challengesinforaging}.

\subsection{Taxonomy of Robot Foraging}
\label{sec:second:taxonomy}

Foraging is a popular problem due to the vast potential opportunities of foraging for application to a variety of real-world problems, from search and rescue \cite{jennings1997cooperative} to toxic waste retrieval to mining. It is beneficial to present a taxonomy of robot foraging that can be used to classify existing research about foraging as well as to contextualize the specific foraging problems addressed by this thesis. One should note that the foraging taxonomy presented also includes non-swarm robotic foraging.

Multiple taxonomies have been developed as robot foraging research has progressed \cite{oster1978caste,ostergaard2001emergent}. However, a more recent and complete foraging taxonomy is presented in \cite{winfield2009foraging}. The taxonomy classifies foraging solutions based on four major axes: the foraging environment, the capabilities and types of robots used for foraging, performance aspects, and the foraging strategy used in a foraging algorithm. Each of these major axes has a number of minor axes which are outlined in Table~\ref{foragingtaxonomytable}.  The following sections discuss each of the major and minor axes in more detail.

\begin{longtable}{ | c | c | c |}
    \caption{Winfield's Robot Foraging Taxonomy}		\label{foragingtaxonomytable} \\
	\hline
	Major Axis & Minor Axis & Value  \\ \hline
	\endfirsthead
	\hline
	Major Axis & Minor Axis & Value  \\ \hline
	\endhead
	\hline
	\endfoot
	\endlastfoot

	\multirow{13}{*}{Environment}
		& \multirow{2}{*}{Search space} 
			& Unbounded \\ \nopagebreak
		& 	& Constrained \\ \cline{2-3}
		& \multirow{3}{*}{Source Areas} 
			& Single limited \\ \nopagebreak
		&	& Single unlimited \\ \nopagebreak
		&	& Multiple \\ \cline{2-3}
		& \multirow{2}{*}{Sinks} 
			& Single \\ \nopagebreak
		&	& Multiple \\ \cline{2-3}
		& \multirow{3}{*}{Object types} 
			& Single static \\ \nopagebreak
		&	& Multiple static \\ \nopagebreak
		&	& Single active \\ \cline{2-3}
		& \multirow{3}{*}{Object placement} 
			& Fixed known locations \\ \nopagebreak
		&	& Uniform Distributions \\ \nopagebreak
		&	& Clustered \\\hline
	\multirow{16}{*}{Robot(s)}
		& \multirow{2}{*}{Number} 
			& Single \\  \nopagebreak
		& 	& Multiple \\ \cline{2-3}
		& \multirow{3}{*}{Type} 
			& Homogenous \\ \nopagebreak
		&	& Heterogenous \\ \cline{2-3}
		& \multirow{2}{*}{Object Sensing} 
			& Limited range \\ \nopagebreak
		&	& Unlimited range\\ \cline{2-3}
		& \multirow{3}{*}{Localization} 
			& None \\
		&	& Relative \\
		&	& Absolute \\ \cline{2-3}
		& \multirow{3}{*}{Communications} 
			& None \\\nopagebreak
		&	& Near \\\nopagebreak
		&	& Infinite \\\cline{2-3}
		& \multirow{3}{*}{Power} 
			& Limited \\\nopagebreak
		&	& Forage \\\nopagebreak 
		&	& Unlimited \\\hline
	\multirow{6}{*}{Performance}
		& \multirow{3}{*}{Time} 
			& Fixed \\  \nopagebreak
		& 	& Minimum \\ \nopagebreak
		& 	& Unlimited \\ \cline{2-3}
		& \multirow{3}{*}{Energy} 
			& Fixed \\ \nopagebreak
		& 	& Minimum \\ \nopagebreak
		&	& Unlimited \\ \hline
	\multirow{18}{*}{Strategy}	
		& \multirow{5}{*}{Search}
			& Limited range \\ \nopagebreak
		&	& Geometrical pattern\\ \nopagebreak
		&	& Trail following\\ \nopagebreak
		&	& Follow other robots\\ \nopagebreak
		&	& In teams\\ \cline{2-3} \nopagebreak
		& \multirow{2}{*}{Grabbing} 
			& Single \\ \nopagebreak
		&	& Cooperative \\ \cline{2-3} \nopagebreak
		& \multirow{2}{*}{Transport}
			& Single \\ \nopagebreak
		&	& Cooperative \\ \cline{2-3} \nopagebreak
		& \multirow{3}{*}{Homing} 
			& Self navigation \\ \nopagebreak
		&	& Home on beacon \\ \nopagebreak
		&	& Follow trail \\\cline{2-3} \nopagebreak
		& \multirow{3}{*}{Recruitment} 
			& None \\ \nopagebreak
		&	& Direct \\ \nopagebreak
		&	& Indirect \\\cline{2-3} \nopagebreak
		& \multirow{3}{*}{Coordination} 
			& None \\ \nopagebreak
		&	& Self-organized \\ \nopagebreak
		&	&  Central control \\ \nopagebreak
		&	& Master slave \\ \hline
\end{longtable}
\subsubsection{Robot Axis}

The robot axis refers to qualities such as the number of robots and the sensory, power, and actuation capabilities of the robots, discussed as follows:

\begin{itemize}
\item \textbf{The number of robots used in the foraging problem}: If only a single robot or few robots are used to forage, then that would not be considered swarm robotics.

\item \textbf{Type}: Robots can either be all identical (homogenous) or have different capabilities (heterogenous). Homogenous swarms are the more common type of robots in swarm foraging, but research using a heterogenous swarm does exist. In particular, the Swarmanoid project uses a heterogenous swarm to forage an item \cite{dorigo2013swarmanoid}. The Swarmanoid swarm consists of three types of robots with three separate responsibilities: the eye-bots locate the item, the foot-bots transport the hand-bot to the item, and the hand-bot climbs until it can grab the item. 

\item \textbf{Object Sensing}: The sensors of the robots can be of a limited range or global sensors with unlimited range. Foraging research that uses global sensors is not considered swarm robotics. Therefore, foraging algorithms using global sensors are not addressed in this literature study.

\item \textbf{Localization} refers to the robot's ability to position themselves in the environment. Localization could be non-existent, relative, or absolute. Absolute localization, such as a global positioning system (GPS), is not considered swarm robotics because the swarm will be dependant on a centralized source of information, violating the swarm robotic principle of decentralization. Relative positioning is the use of local information in order for a robot to position itself in an environment. Relative positioning is suited to swarm robotics foraging problems, since relative positioning is decentralized. Relative localization techniques such as path integration are used in this thesis.

\item \textbf{Communication capabilities} of the robots to each other can have an infinite range, a near range, or not exist at all. Infinite range communication is out of scope for swarm robotics research since local communication is a requirement of swarm robotics. Near range communication can occur via light-based signals \cite{sugawara2002swarming} or local direct communication methods such as Bluetooth. The communication occurs in differing topologies such as broadcast, direct, tree, or graph topologies \cite{dudek1993taxonomy}.

\item Robots can have an infinite power source, a limited power source, or an energy source which can be sustained by foraging resources. Studies focused on energy efficiency or automatic-charging have robots with limited energy sources or an energy source that can be sustained by foraging energy resources \cite{liu2006strategies}.

\end{itemize}

\subsubsection{Performance Axis}
The performance axis characterises robot foraging on the methods used to evaluate the performance of a foraging algorithm, such as energy usage, or the time taken to forage each item. Each minor axis is described as follows:

\begin{itemize}
\item The performance of a foraging algorithm can be classified in terms of the time taken to forage. A foraging experiment can be allowed an infinite time to forage, a fixed time for foraging, or may attempt to minimize the time taken to forage.
\item The energy that a robot can spend on foraging can be a fixed amount of energy, an unlimited amount of energy, or a foraging algorithm can attempt to minimize the energy used to forage \cite{liu2006strategies}.
\end{itemize}

\subsubsection{Strategy Axis}
The strategy axis characterizes algorithms by the techniques that a foraging algorithm implements, such as the type of search used, or the type of recruitment technique used in a foraging algorithm. Each type of strategy is discussed as follows:

\begin{itemize}
\item The first stage of most foraging algorithms is the search for items. There are various techniques for searching for items, such as random search, trail following \cite{schmickl2006trophallaxis}, following other robots \cite{dorigo2013swarmanoid, werger1996robotic}, or groups of robots looking for items together \cite{vincent2004framework}.

\item Once an item is found, it must be picked up. The techniques used by a robot for grabbing an item can be used to classify the algorithm. An item can be grabbed by a single robot or a group of robots can attempt to co-operatively grab an item. Closely related to grabbing an item is the techniqe for transorting an item to the sink. Transport of an item back to the sink can be performed by a single robot or by a a group of co-operating robots. A review of co-operative transport and grabbing techniques is presented in \cite{kube2000cooperative}.

\item In order for a robot to move from its current location to the sink, the robot can employ one of many homing techniques. Examples of homing techniques include beacon homing (where the robot first changes inhibits direction to face a beacon at the sink, and then move towards the beacon), following a trail or odometry \cite{winfield2009towards}.

\item Recruitment has been defined as activities that bring individuals to a location where work is required \cite{holldobler1990ants}. Robots can recruit other robots to forage certain areas. Direct recruitment is where a robot explicitly communicates to another robot to forage an item source \cite{krieger2000ant,labella2006division}. Indirect recruitment occurs when a robot is recruited to forage by inexplicit markers, such as an increase in the number of foragers waiting by the sink, or an increase in the density of other foragers moving towards a particular source of the environment \cite{arkin1992cooperation}. 

\item There exist various techniques for coordination of tasks between the robots in a swarm. A centralized control mechanism can be used for controlling all robots. Centralized control is not considered to be swarm robotics, and thus foraging algorithms that use centralized control are not addressed in this thesis. Master-slave configurations, where one robot leads the others, is another option for coordination of robots. A master-slave configuration with the ability to select a new master in the case that a master is destroyed, is considered as swarm robotics, since the swarm is then robust to individual failures \cite{karpov2015leader, hoeing2007auction}. A swarm with a master-slave configuration that does not have the ability to select a new master is not swarm robotics. Swarms can have self-organized coordination, which means that each robot is in control of its own activities, and the self-organised coordination is an emergent property of the swarm. %TODO citation
\end{itemize}

\subsubsection{Environment Axis}
\label{environmentaxis}
The environment axis is divided into minor axes as follows:
\begin{itemize}
\item \textbf{The type of search space}: The type of search space can be either unbounded or constrained by a boundary wall. Goldberg and Mataric \cite{Goldberg01designand} implement a foraging method for an unbounded environment. Schneider \cite{schneider1998territorial} develops a foraging algorithm where a known constrained environment is split into bounded areas. Each robot is assigned an area to forage.

\item \textbf{Item sources}: Item sources can be of different types and quantities. A single limited source is a source where all items come from the same source and the source has a limited capacity for items \cite{sugawara2002swarming}. A foraging environment with a single unlimited source has one source that an unlimited number of items can be foraged from. A foraging environment can also have multiple sources, such as in \cite{Goldberg01designand}. 

\item \textbf{The number of sinks}: The environment can have a single or multiple sinks. 

\item \textbf{Item distribution}: Item distributions in the environment can vary. Items can be placed  in the environment, in known locations, uniformly distributed, or in clusters.

\item \textbf{Item types}: Items can be classified into types in terms of their activity (active or static), homogeniety (homogenous or heterogenous), and quantity (single or multiple) \cite{Balch99rewardand,
 campo2007efficient, jones2003adaptive}.

\end{itemize}

%%%%%%%%%%%%%%%%%%%%%%%%%%%%%%%%%%%%%%%%%%%%%%%%%
%%%%%%%%%%%%%%%%%%%%%%%%%%%%%%%%%%%%%%%%%%%%%%%%%


\subsection{Nature Inspired Swarm Robotics Foraging Algorithms}
\label{sec:second:natureinspiredalgs}
Foraging is an animal activity performed in order to retrieve food. Social insects, in particular, have found very efficient means of foraging despite their individual simplicity. This section discusses the swarm robotics algorithms that have specifically been inspired by social insects, in particular, those models inspired by ant foraging and bee foraging. 


\subsubsection{Ant Inspired Foraging Algorithms}
\label{sec:second:natureinspired:ant}

Vaughan developed a swarm robotic algorithm allowing for robust transportation of items from a single source to a single sink in an environment with spatial constraints \cite{vaughan2000blazing}. The algorithm presented makes use of both ant-like and bee-like foraging techniques. The ants broadcast the landmarks in the area for odometric localization as well as uses a form of the  honey bee ``waggle dance" that communicates the direction of the food source from the robot that has found the food source to other robots. The robots use path integration, utilized by both ants and bees in order to maintain position and heading estimates. Using these techniques, the robots can communicate multi-segment paths to item sources. The dance communication occurs globally so that every robot knows the multi-segment path to the item source. The path integration technique suffers from accumulation of localization errors over time, which is worsened by global communication, since localization errors experienced by an individual robot get communicated to every single other robot.


Hoff \cite{hoff2010two} classified different types of ant-inspired pheromone-based foraging algorithms by how the pheromones are represented. Pheromones can be represented as physical marks on the environment \cite{fujisawa2008communication} or by sharing data that represents pheromones over wireless networks.

Labella \textit{et al} \cite{labella2006division} proposed a foraging model inspired by the foraging behaviour of ants, where individual robots adapt to the environment using only locally available information, in an attempt to preserve energy. The probability, $p_1$, that a robot will change from rest to searching is increased by a constant when the robot returns to the sink and deposits an item. If a robot fails to forage any item after searching for a specified maximum time, then the foraging attempt is considered a failure, $p_1$ is decremented by a constant, and the robot returns and rests. The study showed an improvement in energy efficiency and the algorithm was able to regulate the number of active foragers. An advantage of this algorithm is its simplicity, since communication is required to regulate the number of robots. The study also showed that there was a global negative effect on overall foraging efficiency at large swarm sizes, despite an increase in energy efficiency.

Hoff \cite{hoff2010two} proposed two ant-based foraging algorithms which do not require marking of the physical environment. Instead, robots themselves become pheromone beacons. The purpose of pheromone beacon robots is to act as pheromone. When many pheromone beacon robots line up, the robots form a pheromone trail that other robots can follow to reach a destination. The algorithm enabled the creation of pheromone trails from the sink to the item source and back.

\subsubsection{Honey Bee Inspired Foraging Algorithms}
Bee swarms have been used in swarm robotics for problems such as path planning \cite{lin2009chaotic}, aggregation \cite{kernbach2009re}, and collective perception \cite{schmickl2007collective}. 

A foraging algorithm was developed in \cite{alers2014biologically} that simulated simplified honey bee recruitment strategies. The algorithm consisted of two phases: an exploratory phase followed by an exploitive phase. Initially, all robots are located around the sink. In the exploratory phase, all robots performed a random search using a L\'evy distribution until an item source was found by a robot. The exploitation phase began when the robot that located the item returned to the sink and communicated the location to other members of the robot swarm using wi-fi. The other robots were recruited and began to commute between the hive and the item source, transporting items back to the sink. This algorithm exhibited three behaviours of bees: waiting around the sink, exploring the environment, and foraging. Unlike bees, the algorithm did not consider the division of labour between foragers, explorers, and waiting robots. The robots used path integration in order to remember the location of the sink source, and used beacon homing in order to locate the sink. The path integration vector was communicated to the other robots using wi-fi, which is used to guide them to the sink source. 
%%%%%%%%%%%%%%%%%%%%%%%%%%%%%%%%%%%%%%%%%%%%%%%%%

\subsection{Challenges in Swarm Robotic Foraging}
\label{challengesinforaging}
%TODO: Citations

Foraging for humans is a near trivial problem. Most children can walk around and retrieve blocks and return them to a single source. The seemingly simple sub-activities required by foraging such as identification and grasping are often relatively complex for robots to perform. Unlike other simpler swarm robotics problems, such as aggregation, foraging is made up of a group of non-trivial sub-tasks. Some of the aspects which are problematic for robots in foraging are discussed below:

\begin{itemize}
\item \textbf{Mechanically challenging interactions}: Despite how far robotics has come as a field, there are still tasks that are trivial for humans, but still very complex for most robots. For instance, the challenge of picking up items \cite{saxena2008robotic} and moving them to another location requires high quality sensors, sensitive actuators, and time-consuming calibration \cite{mondada2005cooperation}. 

\item \textbf{Complex noisy environments}: Although research can be performed in a controlled environment, for robust foraging to be efficiently applied in real-life, environmental factors have to be taken into consideration. Environmental factors, such as light \cite{browning2005real,jungel2003real} and quality of the surface the robots forage on \cite{trianni2006cooperative}, often make a robotic solution complex, non-viable, non-robust, or expensive. To cope with real-life environments, the robots require adaptive algorithms and more complex hardware for their sensors and actuators for basic navigation and object detection in unknown terrains with unknown lighting and weather conditions. 

\item \textbf{Localization, navigation and obstacle avoidance}: Localization is the ability for a robot to depict its position in the environment. In swarm robotics, global knowledge about location (such as GPS) can not be used by a robot to determine its position and where it is going. Non-trivial algorithms must be designed in order for a robot to position itself in the environment as in \cite{zhou2012motion,rothermich2004distributed,arkin1992cooperation}. 

\end{itemize}

Swarm robotics researchers often choose to use simulated robot platforms, such as Stage \cite{vaughan2008massively} or Argos \cite{pinciroli2011argos}, instead of real robots in order to remove or regulate the amount of environmental noise. If a swarm robotics experiment uses real-life robots, the environments are usually simplified by creating an environment where temperature, lighting, and moisture are regulated \cite{labella2006division,nouyan2006group}. Due to the discussed complexities of foraging, existing swarm robotic research often simplifies some of the complexities such as environmental changes, sensor noise, and the use of physical robots in order to focus on a single aspect of foraging. 

%%%%%%%%%%%%%%%%%%%%%%%%%%%%%%%%%%%%%%%%%%%%%%%%%
%%%%%%%%%%%%%%%%%%%%%%%%%%%%%%%%%%%%%%%%%%%%%%%%%
\section{Summary}
\label{foraging:summary}

This chapter reviewed foraging in nature and in swarm robotics. In nature, social insects have efficient emergent foraging behaviours. Ants generally lay pheromone to guide the item search process, but the desert ant does not use pheromone, but instead uses path integration to relocate a food source. Honey bees employ more complex foraging behaviour including division of labour and communication. 

Winfield's foraging taxonomy was presented and discussed so that research can be contextualized in the field. Lastly, existing swarm robotics foraging algorithms that are based on ants and the honey bee were discussed.

\chapter{Division of Labour}
\label{chap:divisionoflabour}


As explained in Section~\ref{flexibility}, division of labour is a strategy that is used by social insects. This chapter defines division of labour and discusses different types of division of labour that occur in social insects. Lastly, division of labour strategies employed by different swarm robotics algorithms are described. 

\section{Definition of Division of Labour}
\label{division:definition}

Oster \textit{et al} \cite{oster1978caste} defined division of labour as a ``stable pattern of variation" of the repertoire of tasks that individuals perform. Each individual specialized on a subset of the complete repertoire of tasks. The subset of the complete repertoire of tasks was called the specialization of an individual and the specialization of individuals varied across the swarm. 

Robinson \textit{et al} \cite{robinson1992regulation}, more simply, defined division of labour as the adjustment of ratios of workers engaged in different tasks based on external and internal stimuli.

There are two types of division of labour in social insects \cite{beshers2001models}: 
\begin{itemize}
	\item \textbf{Temporal polyethism}, where the pattern of tasks being performed by a worker correlates to the age of the worker. In nature, the younger workers may perform tasks within the nest while the older workers perform tasks outside of the nest.
	\item \textbf{Morphological polyethism}, where a worker with extreme physical features in terms of size or shape will specialize more in particular tasks. The more extreme a particular physical feature is, the narrower the repertoire of the worker. For example, soldier ants are larger in order to defend the nest, while the smaller workers are involved with foraging.
\end{itemize}


\section{Models of Division of Labour from Biology}
\label{modelsofdivisionoflabourbiology}
An overview of existing division of labour models from entomology is provided in this section, since insect-like division of labour is used by the algorithms developed in this thesis. The section focuses on common models and models used by the algorithms described in this thesis. Section~\ref{responsethresholdmodel} discusses the response threshold model, while foraging for work is presented in Section~\ref{foragingforwork}. Self-reinforcement models are presented in Section~\ref{selfreinforcement}, while Section~\ref{socialinhibitionmodels} discusses social inhibition models. Finally, Section~\ref{networkmodels} discusses network models for division of labour.

\subsection{Response Threshold Model}
\label{responsethresholdmodel}

The response threshold model gives each individual of a swarm a response threshold for each task that can be performed. The task-specific thresholds vary across the swarm \cite{robinson1989genetic}.

Suppose $R$ is the set of tasks that can be performed by individuals of the swarm. An individual $\gamma$ will only perform task $\nu \in R$ when the environmental stimuli for task $\nu$, $S_\nu$, exceeds the response threshold, $r_{\gamma,\nu}$. Each response threshold for a task, for an individual is constant. If an individual has a lower threshold for a specific task, compared to other individuals in the swarm, then it is likely that the individual will become a specialist in that task \cite{robinson1989genetic}.

Response threshold models have been modelled formally by Page \textit{et al} \cite{page1990self} and Bonabeau \textit{et al} \cite{bonabeau1999role}. Response thresholds have explained temporal polyethism in honey bees where the response thresholds of the honey bees changed as the honey bees aged \cite{robinson1987regulation}.

\subsection{Foraging for Work}
\label{foragingforwork}
Foraging for work assumes that individuals are intrinsically identical and thus performance of individuals at a task is dependent on opportunity to perform a task, rather than intrinsic task preferences \cite{franks1994foraging}.

Foraging for work has two main principles:
\begin{enumerate}
	\item Individuals repeat the same task when possible.
	\item Individuals actively seek work when they have no task.
\end{enumerate}

%This forms the basic model of foraging and also is used in MY honey bee algorithm.

Foraging for work also assumes that tasks are radially spatially localized within the nest. Thus, tasks further away from the nest are performed by older individuals and the younger individuals perform the tasks that are closer to the nest \cite{tofts1993algorithms}.

Foraging for work shows that temporal polyethism does not require age-related differences in the mechanism of task choice and can simply stem from an individual's proximity to the nest. Foraging for work is controversial since it shows that task organization might occur within a social group in absence of selection efforts or an intrinsic mechanism of task performance \cite{franks1994foraging}. 

Foraging for work can be seen as a special case of the threshold model where all individuals have identical thresholds. In foraging for work, temporal polyethism is generated by spatial organisation, where as the response threshold model generates temporal polyethism by differences in internal thresholds. Unlike threshold models, foraging for work does not suffer from inactive individuals as all workers are either performing or seeking tasks \cite{beshers2001models}.

Foraging for work is controversial in the fact that it induces division of labour without any explicit mechanism, but instead as a natural result of the environment \cite{beshers2001models}. However, it has been shown that, in insect colonies, there is intrinsic variation in each individuals' response to environmental  stimuli, rather than from opportunity alone \cite{julian1999undertaking}.

\subsection{Self-Reinforcement Models}
\label{selfreinforcement}

% What are self reinforceme models citation
Self-reinforcement models adjust an individual's probability of performing a task, based on prior success or failure to perform that task. Initially, the probability of performing all tasks are equal. If a task is performed successfully or there is  no opportunity to perform a task, then the probability of performing that task again is reduced. Individuals that continuously successfully perform a specific task become specialists in that task. Task success is directly proportional to the probability of doing the task again \cite{theraulaz1998response, pasteels1987individual}. Self-reinforcement has been attributed to division of labour in many systems in nature, such as ants and bees \cite{spencer1998dynamics}.

\subsection{Social Inhibition Models}
\label{socialinhibitionmodels}
With social inhibition models, individuals change tasks as they age. However, the process of aging is inhibited by social environments \cite{huang1992honeybee}. The model is based on the activator-inhibitor behaviour present in bee swarms. 

In honey bee swarms, juvenile bees work in the hive while older bees forage. When a juvenile bee ages, it transitions from a worker bee into a forager bee. The rate of growth of the juvenile bee is determined by two hormones: the activator hormone and the inhibitor hormone. The activator hormone promotes the growth of the juvenile bee into a forager bee. The activator hormone is released by the juvenile bee. The inhibitor hormone inhibits the growth of a juvenile bee into a forager bee \cite{huang1992honeybee}. The inhibitor hormone is released by forager bees. The inhibitor hormone counteracts the effect of the activator hormone.

The existence of many forager bees in the hive means that inhibitor hormine would be released. Thus, the growth of the juvenile bees would be slowed, and therefore juvenile bees remain worker bees. If forager bees are destroyed, then less inhibitor hormone would be released and more juvenile bees would becomes forager bees, until the number of forager bees have been replenished.

                                
\subsection{Network Models of Task Allocation}
\label{networkmodels}

Network models assume that swarm individuals are identical, and that division of labour is induced by effective communication between individuals of the swarm, about the number of individuals that are required per task \cite{gordon1992parallel}. A number of network models exist \cite{gordon1992parallel,pacala1996effects}.

Network models are similar to foraging for work models in that division of labour is generated by changes in local information encountered by each individual, rather than by intrinsic differences in individuals. 


\section{Division of Labour in Robot Swarms}
\label{background:divisionoflabour:robotswarm}
Division of labour has been used extensively in swarm robotics. This section provides an overview of the use of division of labour in swarm robotics.

Agassounon \textit{et al} \cite{agassounon2002efficiency} designed and demonstrated three response threshold-based methods for division of labour. The robots had to perform a clustering task with multiple object sites. Division of labour techniques were employed to avoid inter-robot interference that occur more often when too many robots attempt to cluster the same site. 

The experiment determined that the robot swarm benefited from threshold division of labour since the robot swarms that employed division of labour performed similarly or better than robot swarms that set a fixed task group size per site. The experiment also determined that local communication improved swarm performance.

Labella \textit{et al} \cite{labella2006division} developed a division of labour strategy for prey retrieval based on Deneubourg's ant self-reinforcement model, discussed in Section~\ref{selfreinforcement}. Experimentation was performed on simulated and real robots. The robots switched from search behaviour to rest behaviour with probability $z$. This probability was incremented by a fixed value when a robot successfully returned to the nest with an item and decremented by the same fixed value when a robot returns to the nest without an item. Labella \textit{et al} classified the foragers into different types, namely, loafers, foragers, or undecided, based on the final value of $z$. Experiments showed that the values of $z$ tended towards extreme values, showing that robots tended to become either loafers or foragers, while only a few became undecided. It was shown that a simple self-reinforcement division of labour model can improve the performance of a swarm of robots at a task, without the need for communication. However, scalability was still a concern, since the swarm experienced negative performance as swarm size increased. 

Liu \textit{et al} \cite{liu2007towards} introduced three mechanisms for adapting the number of resting robots to foraging robots in a simple foraging problem. The mechanisms consisted of variations on how a robot perceives the worker demand. The mechanisms were used to adjust time specific thresholds related to the length of time to wait before returning to work and how long work should be performed for. The mechanisms were as follows:

\begin{itemize}
	\item The internal success of an individual at a specific task.
	\item The collective success of the swarm at a specific task. 
	\item The amount of environmental interference experienced by an individual while performing a task, most importantly the number of collisions with other robots. This technique is a social inhibition model.
\end{itemize}

Four combinations of these mechanism were evaluated as well as compared to a nai\"ve foraging approach. The performance of the swarm was measured as the total energy spent by the robot swarm. The efficiency was evaluated over different robot quantities and food densities.

The experiments discovered that the use of the mechanisms improved performance significantly compared to swarms without such mechanisms. It was also shown that the mechanisms resulted in emergent division of labour, regulating the number of foraging and resting robots. The experiments resulted in robustness in environmental changes related to the density of items. The experiments using collective success of the swarm achieved the highest net energy income to the swarm and also had the fastest adaptation of the ratio of foragers to resting robots when food density changes. Liu \textit{et al} \cite{liu2007towards} noted that scalability was not tested in the experiments.

The majority of research in division of labour for swarm robotics was concerned with regulating the number of active robots in a swarm. The ability of a swarm to adjust the number of individuals actively participating in a task with the number of robots at rest enabled robot swarms to adapt to changes in the environment. Existing research also concluded that division of labour improved swarm robustness, by replenishing the number of active robots when active robots are destroyed. 

Existing research did not test scalability of the division of labour techniques, since the maximum swarm size was usually ten or less individuals. This thesis will evaluate the scalability of the division of labour techniques used. There exists scope for investigating other types of division of labour in swarm robotics, such as network models, but this is not the scope of this thesis. 


\section{Summary}
\label{sec:second:summary}
Division of labour is the adjustment of ratios of workers engaged in different tasks, based on external or internal stimuli. Division of labour is used in social insect societies such as bees and ants. Multiple models for division of labour have been explored by biologists. Extensive research has been performed about response threshold models, foraging for work, network models and self-reinforcement models. Despite the large number of models, little verification of these models have been performed in real or simulated environments. 

The use of division of labour in swarm robotics was reviewed. Most of the reviewed research focused around the problem of using division of labour to regulate the number of active robots to the number of inactive robots. The techniques used in swarm robotics employ a variety of division of labour models such as response threshold models, self-reinforcement, and social inhibition models. 

The author notes that the reviewed research did not adequately address the scalability of the division of labour techniques.
