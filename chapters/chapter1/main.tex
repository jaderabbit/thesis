%%%%%%%%%%%%%%%%%%%%%%%%%%%%%%%%%%%%%%%%%%%%%%%%%
%%%%%%%%%%%%%%%%%%%%%%%%%%%%%%%%%%%%%%%%%%%%%%%%%

\chapter{Swarm Robotics}
\label{chap:first}

%%%%%%%%%%%%%%%%%%%%%%%%%%%%%%%%%%%%%%%%%%%%%%%%%
%%%%%%%%%%%%%%%%%%%%%%%%%%%%%%%%%%%%%%%%%%%%%%%%%

This chapter provides a broad view of swarm robotics in terms of its origin, motivation, development and current state.
Section~\ref{sec:first:definitionswarmrobotics} defines the concept of swarm robotics while the origin and a brief history of swarm robotics is provided in Section~\ref{history}. Motivations for swarm robotics are outlined in Section~\ref{motivations}, followed by Section~\ref{currentstate} which expands on the current state of swarm robotics. Lastly, the chapter addresses the challenges faced by swarm robotics in Section~\ref{challenges}.

%%%%%%%%%%%%%%%%%%%%%%%%%%%%%%%%%%%%%%%%%%%%%%%%%
%%%%%%%%%%%%%%%%%%%%%%%%%%%%%%%%%%%%%%%%%%%%%%%%%

\section{What is Swarm Robotics?}
\label{sec:first:definitionswarmrobotics}

Swarm robotics is the study of the co-coordination of large numbers of relatively simple robots in order to perform a single function, without the existence of central control. Swarm robotics focuses on how to create robotic algorithms that result in the emergence of a desired complex behaviour \cite{csahin2005swarm}.

Swarm robotics typically draws inspiration from the observation of social insects, for examples, ants \cite{hoff2010two}, cockroaches \cite{garnier2005aggregation} and bees \cite{lee2012foraging} who exhibit collective behaviour in the growth and maintenance of their societies\cite{bailishive, wilson1971insect}. The more interesting aspects of insect societies such as foraging and collective transport can be studied and modeled in order to achieve similar behaviour in a group of simple stimulus-response robots. Swarm robotics also draws on concepts from other societies such as the amoeba's aggregation into slime \cite{schmickl2007navigation} and communication, propulsion and sensing in bacteria \cite{dhariwal2004bacterium,martel2010using}. Social insect societies exhibit desired qualities of a robot swarm namely robustness, scalability and flexibility.

\section{Motivations for Swarm Robotics}
\label{motivations}

Swarm robots draws inspiration from insect swarms due to the characteristics of robustness, flexibility, scalability and to a lesser extent cost-effectiveness. The following sections describe the characteristics of insect swarms that all swarm robotics algorithms strive to achieve. 

\subsection{Robustness}
\label{robustness}


In swarm robotics, robustness is defined as the ability to continue to function despite failures or abnormalities of the respective individuals and environments. Three aspects of insect swarm algorithms have been identified as enabling robustness: redundancy, decentralized coordination, and multiplicity of sensing \cite{csahin2005swarm}.

Swarm robotics algorithms achieve redundancy by giving all or a portion of the robots the same capabilities. In this way, if a percentage of the swarm malfunctions, the other robots have the capability to take the malfunctioning robots' place. The loss of a single individual is compensated by another.

Decentralized coordination can be attained by creating algorithms that do not depend on the life span of any single individual or a few individuals of the swarm.

The use of a large number of individuals increases the total number of sensors.  As a result of sensory multiplicity, the total signal-to-noise ratio is increased. If the sensor data of the robot swarm is adequately aggregated by the swarm, the overall effect of noise can be decreased or eliminated. 

\subsection{Flexibility}
\label{flexibility}

A swarm robotics algorithm should exhibit the ability to adapt and adjust to new or changing requirements. Ants and bees do this by having effective division of labour strategies. The individuals in many insect societies can take on a variety of different roles required by the nest such as brooding, foraging or nest maintenance due to changes in the environment \cite{morley1946division}. The ability to take on different roles when requirements change is known as division of labour \cite{beshers2001models}. Many swarm robotics algorithms make use of social insect-like division of labour in order to adapt to changing requirements  \cite{gerkey2004formal, labella2006division, liu2007towards}. The use of division of labour assists the swarm in maintaining flexibility.

\subsection{Scalability}
Scalability refers to the ability of the robotic swarm to expand the self-organizational mechanism for larger problems easily by simply adding more robots to the swarm. The performance of the algorithm should not be negatively effected by an increase in swarm size, and an increase in swarm size should adequately improve the performance of the swarm. In order for an algorithm to be considered as a swarm robotics algorithm, scalability studies should be performed in order to determine whether an algorithm's performance will increase or decrease at an adequate rate as the swarm size increases. Scalability studies have been performed in \cite{bahgecci2005evolving,nouyan2008path,zarzhitsky2005distributed}

\subsection{Cost-effectiveness}
The idea that swarm robotics is more cost effective than traditional robotics runs off the premise that multiple inexpensive robots are likely to be cheaper to buy and maintain than a single, large, more complex robot. Cost-effectiveness is a debatable attribute, which depends on the complexity of the robotics required. 


\section{Characteristics of Swarm Robotics}
\label{characteristics}

Drawing from this inspiration from social insects, swarm robotics algorithms have a number of characteristics that distinguish them from other robotics algorithms. The characteristics are as follows:

\begin{enumerate}

\item \textbf{Quantity}: Studies are regularly focused with scalability as a potential characteristic of swarm robotics algorithms even if real-robot experimentation is limited to only a few individuals. In order to test the scalability of an algorithm, models or simulations are often built for experimentation purposes. Erol Sahin proposes that 10 individuals are a reasonable lower bound for a group of robots to be considered a swarm \cite{csahin2005swarm}. A group of robots with less than 10 robots, is considered just that - a group of robots. 

\item \textbf{Homogeneity}: A group of robots that has ``too many" individuals with unique characteristics, is no longer considered a swarm, since it would likely violate the robustness requirements. Robustness would be violated since loosing a specific robot that is the only individual of its kind capable of performing a specific function would mean that the swarm no longer can perform that function and is therefore, no longer robust. It is of the author's opinion that heterogeneity can exist in robot swarm provided that the redundancy of each type of robot in the swarm is high enough, or if the failure of that specific robot can be compensated for, even if only to a lesser efficiency, by the remaining types of robots. The evaluation of the degree of homogeneity of a swarm of robots has been discussed and a potential measure of homogeneity is proposed in \cite{balch2000hierarchic}.

\item \textbf{Decentralization and Autonomy}: There should be no single point of failure in the group of robots and the robots must have complete self-control of their actuation and sensors, without central control. Decentralization and autonomy of the robots are key to the robustness of the system. If a particular process requires a single leader to make a decision, such a process should include the ability to detect situations when a leader is no longer present (due to malfunction or destruction) and then to re-elect a new leader without human intervention.

\item \textbf{Localization}: Assume a group of robots that are dependant on global sensors and communication - for example, all inter-robot communications have to pass through a centralized server, or an overhead camera that all individuals of the swarm connect to and use for navigation. The problem with global sensing or communication is that the swarm has a single point of failure, thus violating decentralization. Thus, local sensory and communication abilities are required in order to uphold the decentralization requirement. If a single robots local communication or sensors fail, then the rest of the swarm are unaffected.

\item \textbf{Simplicity}: A single robot should be under-equipped to handle the task by itself, however, collaboration amongst a group of the same robots should assist the completion of the task. In the author's opinion, the level of simplicity of robots is a topic worth debating since ``simple" is a relative term and various complexities of robots have been used in swarms. For example, the kilobot project \cite{rubenstein2012kilobot} has built very low cost simple robots to enable testing of collective behaviours on very large swarms. The kilobots do not even have wheels for motion but simply have a three rigid legs with vibration motors for actuation. Sensors are limited in the form of a simple ambient light sensor and an infrared transmitter and receiver, and coloured LED lights for communication. 

On the other side of the spectrum, Melliner \textit{et al} \cite{kushleyev2013towards, mellinger2013cooperative} explore robotics algorithms for swarms of relatively advanced, expensive quadcoptors with a variety of top-class sensors such as magnetometers, accelerometers, gyros, barometer for altitude sensing and two Zigbee transceivers for communication.
\end{enumerate}

 The field of swarm robotics has been applied to a large variety of problems such as search and rescue \cite{mondada2002search}, item or garbage collection \cite{balch1995io}, autonomous inspection of machinery \cite{correll2007challenging}, and military formations for military application \cite{balch1998behavior}.


%%%%%%%%%%%%%%%%%%%%%%%%%%%%%%%%%%%%%%%%%%%%%%%%%
%%%%%%%%%%%%%%%%%%%%%%%%%%%%%%%%%%%%%%%%%%%%%%%%%

\section{History}
\label{history}
 
%TODO: Turn into an introduction. 
Swarm robotics is simply the latest buzzword for a concept that has been of interest since the 1980s and has gone by many other names in the past. Early swarm robotics research explored the use of robots as a tool with which to model and validate entomology research on social insects \cite{dorigo2014swarm, beni1993swarm, seeley2009wisdom}.

This section aims to briefly highlight the history of swarm robotics and it's progress by addressing behaviour based robotics, the origins of multi-robot systems, and other early research.

\subsection{Journey from Classic Robotics}
\label{journeyfromtraditionalAI}

It is plausible that the first step in the movement from symbol-based classical robotics to swarm robotics was the movement from complex symbolic systems towards more simpler architectures which began in the mid-to-late 1980s. In Rodney A. Brooks' iconic paper ``Elephants don't play chess" \cite{brooks1990elephants}, the movement from traditional artificial intelligence towards what they brand ``nouvelle" artificial intelligence is discussed. Brooks explains that traditional symbol-based classical artificial intelligence made certain assumptions about how intelligence worked and those assumptions actually impeded that form of artificial intelligence. 

Brooks presents the physical grounding hypothesis which is the idea that intelligence is composed of individual modules that generate behaviour and the co-existence and co-operation of these modules allow for the emergence of complex behaviour. The physical grounding hypothesis states that the world is its own best model and thus in order to accurately make decisions about the real world, one should add sensing and actuating capabilities to artificial intelligence agents.

On the other hand, classical or symbolic artificial intelligence (most notably used by systems such as Prolog) makes use of the symbol system hypothesis that states that all intelligence operates using symbols. The implication that symbols represent all world entities fails since symbols may not be up to date with the environment or have the ability to represent unseen things. 

The physical grounding hypothesis lead to the development of behaviour-based robotics and more concretely the subsumption architecture \cite{brooks1986robust} by tightly connecting perception to action. A subsumption program organises finite state machines into layered incremental networks. 

Multiple systems were developed using the subsumption architecture with great success. Allen, a robot that has three layers of behaviours: i.e. obstacle avoidance layer, a random wandering layer, and a search for the furthest point layer. These 3 separate behaviours cause the robot to adequately explore an environment \cite{brooks1986robust}.

Herbert, the soda can collector robot, was a notable development. Herbert was required to complete the significantly more complex task of search for and picking up empty soda cans from people's desks \cite{connell1989colony}. To solve the problem, 15 individual behaviours were developed, with no communication between the behaviour generating modules. Other subsumption implementations include Toto \cite{mataric1990distributed}, Squirt \cite{flynn1989world} and Genghis \cite{brooks1989robot}.

The key feature that can be determined from the success of physically grounded systems is that interactions of simpler non-goal directed behaviour results in emergence of goal-directed behaviour. Ronald Arkin discusses the idea that behaviour-based robotics imposes a biologically bottom up approach with the need that intelligence must be reactive to the dynamic environment and that intelligence needs to generate robust results despite noisy complex real world environments \cite{arkin1990integrating}.

Behaviour-based robotics is the idea that complex robot behaviour can emerge from a combination of simple behaviours. It is of the author's opinion that behaviour-based robotics signifies the a change in focus in research in robotics from complex indviduals robots, to complex robots composed of simple behaviours (behaviour-based robotics), and finally to multi-robot systems consisting of simple robots composed of simple behaviours. Most work in swarm robotics began after the introduction of the behaviour-based robotics paradigm \cite{arai2002editorial}.

\subsection{Evolution of the term ``Swarm Robotics" and early research}
\label{early-research}
%TODO: Make flow
%How to rearrange? Potentially remove?

Early research involving collaboration of multiple agents was originally a way of verifying entomology research on social insects \cite{dorigo2014swarm, beni1993swarm, seeley2009wisdom}. Many researchers modelled and simulated aspects of the insect colonies that were being studied in order to learn more about their self-organizational mechanisms. 

Early work included modeling the excavation behaviour and tunnel creation of ants in order to simulate the rate of excavation \cite{sudd1975model}. Agent simulation was used to validate a model of the spatial arrangement and diet overlap between colonies of desert ants \cite{ryti1984spatial}.

Seeley \textit{et al }\cite{seeley1991collective} formulated a mathematical model of collective decision-making in bee colonies where digital simulations were used to determine validity of a model of collective foraging in bees based on individual behaviour rules \cite{de1998modelling}. The foraging behaviour of ants has also been simulated in early research \cite{lopez1987optimal}. Such studies discovered many of the features about social insects that are exploited in swarm robotics today.

From the late 1980s until the mid 1990s, swarm robotics research emerged under many different guises: Collective robotics \cite{kube1993collective}, cellular robotics \cite{freund1984design}, co-operative mobile robotics \cite{cao1997cooperative}, distributed robotics \cite{asama2013distributed}, and multi-robot systems \cite{mataric1995cooperative}. It is difficult to determine which terms were used first as they seem to have been in use concurrently. Essentially, all of these terms have converged to what we now know as swarm robotics.

In one of the earlier, more notable papers, Freund explores the design of the structure of multi-robot systems based on nonlinear control approaches. Freund demonstrates the design approach on the collision avoidance of two robots working on the same space \cite{freund1984design,freund1986pathfinding}. Around the same time, the multi-robot design paradigm ACTRESS was developed to also address the design of autonomous distributed robot systems \cite{asama1989design}. 

In the late 1980s and early 1990s, Fukuda et al \cite{fukuda1989communication, fukuda1990analysis} introduced work in the field of cellular or configurable robotics. Cellular robotics was defined as a robotic system that can reconfigure itself based on dynamic environmental requirements. The idea is that cells of smaller, simpler robots can dock with each other to form a single, compound structure that can more effectively handle the pressures of the new environment. The research predominantly explored how the distributed communication between the cells would work. Fukuda \textit{et al.} evaluated and described their CEBOT structure with an optimal knowledge allocation method to communicate between cells. Around the same time, Beni \textit{et al} \cite{beni1991theoretical} explored which problems would have needed to be solved for cellular robotics to become feasible in real-world application.

The various ideas for the use of multiple robots to a achieve a task have now been grouped under the umbrella term of swarm robotics. Since the 1990s, swarm robotics has become a popular field of research in artificial intelligence and Swarm robotics has peaked the interest of the world. 


\section{Current State of Swarm Robotics}
\label{currentstate}

There are a number of axes of focus on swarm robotics research. The core dimensions of the field are modelling, behaviour design, communication, analytical studies, and applications. This section discusses those core dimensions in order to give an overview of the current state of swarm robotics research. 

\subsection{Swarm Robot Analysis}
Macroscopic and microscopic models are often used to determine to what extent a property, such as scalability or performance, is satisfied or not satisfied. Microscopic models aims to model each robot individually where as macroscopic models model the entire swarm and are modelled in formal mathematics.

\subsubsection{Microscopic Models}
\label{microscopicmodels}

Microscopic models are concerned with individual agents and analyse interactions between each robot and between a robot and the robot's environment. Microscopic models are implemented to varying levels of detail - some are simplified to a 2D grid world environment where as others choose to opt for a full 3D environment with dynamic physics. The level of detail is dependant on the problem and what is being researched.

Predominantly, microscopic models take the form of simulators and are used to validate swarm robotics systems. Swarm-robotic specific simulators include Stage \cite{vaughan2008massively} and ARGoS \cite{pinciroli2011argos}, both of which focus on simulating a large number of agents. A concern of obtaining and maintaining a large number of agents is one of the reasons why simulators are used in swarm robotics instead of real world experimentation.

\subsubsection{Macroscopic Models}
\label{macroscopicmodels}

Macroscopic models are focused on a higher level view of the entire system and the individuals of the system are not analysed. A variety of macroscopic models exist as follows: 
\begin{itemize}
	\item \textbf{Rate and differential equations}: Rate equations are used to describe the change in the proportion of robots in a particular state over time. Rate equations were used to model a variety of swarm robotics problems such as  clustering \cite{martinoli1999understanding}, stick pulling \cite{lerman2001macroscopic}, foraging \cite{lerman2002mathematical}, chain formation \cite{trianni2002modeling}, and multi-foraging \cite{campo2007efficient}. However, modelling space and time is complex since robots' positions in space are not modelled explicitly.

Similarly, differential equations have been used to model swarms as well as factors such as noise, stochasticity and spatiality. Unfortunately, differential equations are often computationally expensive and complex to solve \cite{hamann2008framework, prorok2011multi}

	\item \textbf{Classical control and stability theory} has been used to prove swarm properties \cite{gazi2005stability,liu2004stable, schwager2011time}.  Classical control methods have the advantage of being based on sound mathematics. However, they often rely on assumptions in order to simplify the modelling process. In reality, many of those assumptions are continuously violated.
	
	\item \textbf{Other methods}
	Many other mathematical modeling approaches have been used in a swarm robotics context, such as linear time temporal logic to define safety and liveness of swarm individuals \cite{winfield2005formal}, probabilistic model checking to verify swarm properties \cite{konur2012analysing}, or using branching processes to model communication of a swarm of aerial robots \cite{mathews2010establishing}. 
\end{itemize}

\subsubsection{Real-robot analysis}

Since it is implausible to attempt to simulate reality completely, real robot experimentation is integral to validating the behaviour of the swarm. Real-robot simulations usually occur in controlled environments that have the ability to control the level of noise and environmental disruption. 
The disadvantage of performing real robot analysis is that experimentation is generally more costly, complex, and time consuming than simulation or modeling methods. The purpose of real robot analysis is to show that the prospective swarm behaviour is actually obtainable \cite{brambilla2013swarm}.

\subsection{Behaviour Design}

In order to achieve emergent behaviours, a number of approaches for the design of robot controllers have been used. This chapter addresses the techniques of behaviour design, namely non-adaptive, learning, and evolutionary approaches.

\subsubsection{Non-adaptive design}
Non-adaptive behaviour design, in general, refers to techniques of behaviour design where by the algorithms have specifically been hand-crafted by the human designer. These techniques either utilize mathematical approaches or focus on how to combine simpler behaviours to achieve the desired emergent behaviour. 

\begin{itemize}
	\item \textbf{Subsumption Architecture} - Subsumption architecture is a robot architecture developed to aid the construction of robots that can interpret and respond to multiple environmental stimuli efficiently and correctly. In subsumption architecture, each robot behaviour forms a separate module that can inhibit other behaviours \cite{connell1989colony}. These modules are arranged in a series of incremental layers connecting perception to action. 
	
	\item \textbf{Probabilistic Finite State Automata}: Probabilistic finite state automata are a method to represent dynamical systems with finite state spaces. Each behaviour is a state and transistions between these states occur at specified probabilities based on external input \cite{labella2004efficiency, soysal2005probabilistic}. The algorithms addressed in this thesis take the form of probabilistic finite state automata.  
	
	\item \textbf{Distributed Potential Field Methods}: In physics, potential energy is the energy possessed by a body resulting from position or configuration. For instance, a robot that is on top of a slope has greater potential energy than a robot at the bottom of the slope. A potential field is a collection of vectors that representing the direction and force of the potential energy - i.e. the directions that a robot has the potential to move. Thus each robot has a potential field which is made up of a vector combination of individual behaviours (for instance navigation behaviour resulting in a movement vector in one direction would be added to the vector from obstacle avoidance behaviour). A distributed potential field refers to the potential field of one robot to another robot in a robot swarm. Distributed potential fields are useful in building swarm behaviours that require spatial distribution between robots such as pattern formation, obstacle avoidance or motion coordination. Examples of using distributed potential field methods can be seen in \cite{bennet2010distributed, barnes2007unmanned, kim2006decentralized}  
\end{itemize}

\subsubsection{Learning}

Robots can have the ability to learn behaviours suited to solve certain problems. A number of techniques have been used to learn behaviours, most notably reinforcement learning and neural networks \cite{samejima1999adaptive, sun1999multi}

\subsubsection{Evolution}
Neural network or tree-based controllers for swarms of robots can be evolved or optimized using a variety of algorithms from swarm to genetic techniques \cite{baldassarre2003evolving, tuci2014evolutionary}. Francesca \textit{et al} define two evolutionary approaches, AutoMoDe-Vanilla and EvoStick \cite{francesca2014automode,francesca2014experiment}. The experiments compared the effectiveness of evolving swarm robot controllers in comparison to human created controllers. The interesting result is that the evolved AutoMoDe-Vanilla controller was able to outperform the human created controller, thus showing the promise for using evolutionary techniques to design swarm controllers.

\subsection{Interactions}
A key element of swarm robotics is the interactions between robots in a swarm. Cao \textit{et al} \cite{cao1997cooperative} classified the methods of information transfer between robots in their  survey on swarm robotics. Interactions were segmented into three types:
\begin{itemize}
	\item \textbf{Interaction via sensing}, where no direct communication between robots occurs. Robots simply obtain information from their senses, for instance the stick pulling problem whereby a robot can sense another robot pulling the stick \cite{ijspeert2001collaboration}. 
	\item \textbf{Interaction via the environment}, where the robots modify the environment in some way. Other robots then perceive and understand that modification such as pheromones in ants. Robots have mimicked pheromone-like deposits in a number of ways, such as a substance distributor \cite{fujisawa2014designing} or beacon deployer\cite{barth2003dynamic}.
	\item \textbf{Interaction via communication}- where robots interact and transfer information by assigning meaning to specific signals or having a specific language \cite{hoff2010two}. The language has been in the form of light signals or even direct data transfer via bluetooth.  
\end{itemize}

\subsection{Behaviours}
\label{swarmrobotapplications}

Brambilla \textit{et al} \cite{brambilla2013swarm} present a taxonomy of swarm behaviours. These behaviours can be used as ingredients to solve more complex problems.

\subsubsection{Spatially Organizing Behaviours}
The following behaviours involve agents positioning themselves in their environment in relation to the position of other agents:

\begin{itemize}
	\item \textbf{Aggregation} Aggregation is the movement of individuals towards one another to form a cluster and has become one of the fundamental swarm behaviours. In nature, aggregation aids protection from predators, the ability to defend against an unfavourable environment and locate partners \cite{bonabeau2001self}. A variety of approaches to swarm robot aggregation have been investigated \cite{yan2011algorithm, soysal2007aggregation, trianni2003evolving}. A more recent approach finds inspiration in honey bees \cite{schmickl2011beeclust, schmickl2009two}
	
	\item \textbf{Pattern formation}:
	The goal of pattern formation is for deployed robots to maintain specific distances between each other in order to create a particular shape. Implementation usually utilizes virtual forces in order to co-ordinate positioning of the agents. A review can be found in \cite{bahceci2003review, hettiarachchi2009review}.
	
	\item \textbf{Path formation} is the creation of chains of robots in order to connect two or more points. The robot chains can serve as pheromone paths in order to navigate other robots through environments, because generally, robots are not equipped with pheromone sensors and deployers. Research in path formation forms part of environmental exploration and navigation \cite{nouyan2006chain}. 
	
	Path formation is useful as a navigational strategy when dealing with robot swarms since traditional complex forms of navigation do not scale well as the number of robots increases. Path formation techniques also have the advantage of being more failure tolerant since they do not rely on a specific individual.
	
	Research in path formation initially had the robots emit a signal communicating their position. Unfortunately, this introduces the problem of global localiation \cite{goss1992harvesting}. Later, path formation using real robots in a prey retrieval experiment, where the robots used physical contact to sense each other, was studied \cite{werger1996robotic}. More recent approaches attempt to give directionality to the chains by giving the chains a cyclic directional pattern. The approach was tested with real robots to transport heavy objects \cite{nouyan2006group}. Path formation has been used to connect two objects that are too far from each other to be perceived at the same time by a robot \cite{nouyan2006chain}.

	\item \textbf{Self-assembly and morphogenesis}: 
	Ants can create connections to each other to pull large prey and build bridges or walls. Self-assembly refers to individual robots with the ability to connect in order to create a compound structure of robots. Self-assembly was researched very early in the swarm robotics field, originally known as cellular robotics. In application, self-assembly has be used to stabilize robots on difficult terrains, and to combine forces to transporten another object \cite{brambilla2013swarm}. Self-assembly has remained one of the more complex behaviours, due to the challenge of morphogenesis and how to control the structure in a consistent manner. Reviews on self-assembly are presented by Gro{\ss} and Dorigo \cite{gross2008self}.
	
	\item \textbf{Object clustering and Assembling}:
	The goal of clustering is to group similar objects together, and the goal of assembling is to link objects in a required way. Challenges in the field of clustering and assembling are to do with interference from robots and obstacles near the cluster site. A variety of implementations of clustering have been developed, such as \cite{beckers1994local}, wall creation \cite{wawerla2002collective} and 2D and 3D structure creation \cite{werfel2006extended, werfel2011distributed}. 
\end{itemize}

\subsubsection{Navigation behaviours}
Robot navigation is difficult task and navigational complexity will increase as the number of robots increases due to the increase in inter-robot interference.  In order to decrease interference, knowledge can be gained by sharing information about navigation between robots. Navigation behaviours are outlined in the following section:

\begin{itemize}
\item \textbf{Collective exploration}: Collective exploration is the use of a swarm to search and map out features of an area. Brambilla \textit{et al} \cite{brambilla2013swarm} break collective exploration into two behaviors, namely area coverage and swarm-guided navigation. The goal of area coverage is to deploy robots with the aim of creating a grid of communicating robots over a space. Area coverage is usually approached by using virtual physical forces\cite{howard2002mobile,stirling2010energy}.
Swarm-guided navigation is simply to navigate an environment using swarms. Swarm navigation is usually achieved using probabilistic finite state machines \cite{payton2001pheromone,ducatelle2011self}. 

\item \textbf{Coordinated motion} 
enables robots to move together like a swarm of locusts, a school of fish, or a flock of birds. Navigating effectively as a swarm reduces interference between agents, increases the safety of individuals and reduces energy consumption \cite{parrish2002self}. Research performed in this domain is focused around maintaining a constant distance between each robot as well as on a means of calculating the overall heading direction of the swarm \cite{turgut2008self,ferrante2010flocking,baldassarre2003evolving}.

\item \textbf{Collective transport} has become a benchmark for swarm robotics since it requires numerous collaborative aspects such as task allocation, conflict resolution, and communication. Mataric \textit{et al} use a simple communication protocol on real robots to compensate for noise-ridden sensing to use a group of robots to move a box \cite{mataric1995cooperative}. Hiroshi \textit{et al.} do not utilize explicit communication, but rather allow robots to infer the intention of other robots via their behaviour, in order to enable the robots to correctly place desks in an environment. The study indicates the benefit of collaboration \cite{sugie1995placing}. Gerkey \textit{et al} implement a dynamic task allocation algorithm for a robot with an auction-based mechanism using publish/subscribe communication also to perform a box pushing exercise \cite{gerkey2002sold}. The SWARM-BOTS project involved substantial work with the s-bots. S-bot robots have the ability to link to each other. Experimentation showed that the chained s-bots were able to transport items more effectively \cite{gross2004group, dorigo2005swarm, ferrante2013socially}.

\end{itemize}
		
\subsubsection{Collective decision-making}
With any multi-agent system, the ability of the system to collectively make decisions results in a far more complicated system than traditional robotic systems. This section discusses and provides examples of types of collective decision making behaviours. 

\begin{itemize}
	\item \textbf{Consensus achievement} is the process of how a group of robots come to a decision among a variety of alternative choices in order to maximise system performance. This is observed in ant colonies to determine the shortest path \cite{bonabeau2001self} and is used by bees decide on the next best site for a nest \cite{seeley2001nest}.
	
	\item \textbf{Division of Labour} - occurs naturally in insect colonies \cite{gautrais2002emergent}, most notably in bees and ants. Division of labour is a key primitive problem in swarm robotics. Most tasks require different behaviours in order to reach completion. The goal of division of labour in swarm robotics is to most effectively assign behaviours to robots in the swarm in order to more optimally achieve some final goal. Division of labour is usually addressed as part of a more complex problem such as foraging where by multiple roles are required for efficient foraging to occur such as foragers and scouts and the ratio of foragers to scouts. Another common division of labour problem is to find and maintain the optimal number of robots to perform an activity as opposed to resting. Too many robots performing a task may increase the number of collisions and increased collisions decreases the overall efficiency of the swarm. In the same line, too many robots performing a task leads to wasting of energy, thus the need for effective division of labour between active and passive robots arises \cite{jones2003adaptive, krieger2000call}.

\end{itemize}

\section{Challenges}
\label{challenges}

Sahin \textit{et al.} suggest a number of challenges faced by swarm robotics \cite{csahin2008special}. These challenges are discussed below: 

\begin{itemize}

\item \textbf{Emergent Algorithm Design}: As the field stands, there is no real way to actually design individual behaviour in order to guarantee a specific emergent behaviour. Hamman \textit{et al.} attempt to devise a mathematical model to assist in modelling swarm behaviour and communication, to help bridge the local behaviour to global behaviour \cite{hamann2008framework}. Unfortunately, Hamman \textit{et al.}'s model is limited to specific sensory or actuation abilities. 

\item \textbf{Experimentation}: In order to properly study swarm robotics, the appropriate infrastructure is required - from robot hardware, to cameras with which to monitor the actions of the robots. The need for real robots can be alleviated by use of simulation environments. A number of simulators have been developed in order to enable easier experimentation at different levels of accuracy \cite{vaughan2008massively,pinciroli2011argos, luke2005mason, bredeche2013roborobo}. Some are only 2D grid-world models while others attempt to mimic real life as accurately as possible \cite{michel1998webots}. It is of the author's opinion that the field suffers from a lack of real robot experimentation and that little validation of the accuracy of simulators has been performed. 

\item \textbf{Analysis and Modelling}: Swarm robotic systems are stochastic non-linear systems. It is difficult to prove properties as well as optimize any required parameters of stochastic non-linear systems. Thus models for swarm robotics have to make assumptions about the swarm, or the environment, to simplify the swarm robotics system so that it is easier to prove specific properties of the swarm or to optimize swarm parameters. The risk of the assumptions is that the conclusions achieved from analysis of the models, may not reflect reality and little work in verification of .

\item \textbf{Robotic systems are challenging in themselves}. Swarm robotics will also still be weighed down by the challenges faced by traditional robotics. Even if the swarm element is removed from the system, robotic systems requires expertise from a variety of very different fields such as physics, programming, artificial intelligence, and mechanics.
\end{itemize}

The study of the process of how to achieve reliable real-world swarm systems and how to address the above challenges, is known as swarm engineering \cite{brambilla2013swarm}. %TODO: Cite some real world systems. 

\section{Summary}
\label{sec:first:summary}

%%%%%%%%%%%%%%%%%%%%%%%%%%%%%%%%%%%%%%%%%%%%%%%%%
This chapter defines swarm robotics as the study of the co-coordination of large numbers of relatively simple robots in order to achieve a single goal, without the existence of central control. The history of swarm robotics was traced from its potential origins in classical robotics. The guises that swarm robotics went by in its early days were discussed and consolidated into a single concept. Motivations for swarm robotics were discussed in order to contextualize the relevance of swarm robotics and its recent popularity. This chapter addressed the current state of swarm robotics. The current state of swarm robotics was broken down into sub-fields of study, namely, analysis, behaviour design techniques, interaction types, and the many swarm behaviours. Finally, the challenges that swarm robotics faced were highlighted. The next chapter discusses foraging in nature and swarm robotics.