%%%%%%%%%%%%%%%%%%%%%%%%%%%%%%%%%%%%%%%%%%%%%%%%%
%%%%%%%%%%%%%%%%%%%%%%%%%%%%%%%%%%%%%%%%%%%%%%%%%

\chapter{Conclusions}
\label{chap:conclusions}

This chapter provides an overview of the conclusions arrived at in this work. For ease of reference, the research objectives which were given at the outset of this thesis are reiterated in Section~\ref{sec:introduction:objectives}. A synopsis of the contributions and experimental findings of this thesis are presented in Section~\ref{sec:conclusions:conclusion_summary}, while Section~\ref{sec:conclusions:future_work} outlines possible future work that can be performed.


\section{Objectives}
\label{sec:introduction:objectives}

The primary objectives of this thesis are as follows: To

\begin{itemize}
	\item conduct a survey of the swarm robotics field;
	\item conduct a survey of foraging in social insects and  swarm robotics;
	\item define the prioritized foraging problem;
	\item propose metrics for evaluating the performance of swarm robotics algorithms on the prioritized foraging problem;
	\item propose and develop different nature-inspired algorithms in a simulated swarm robotic environment, to be evaluated on the prioritized foraging problem; and
	\item evaluate the efficiency, flexibility, scalability, and robustness of the nature-inspired algorithms over different environments and different swarm configurations, on the prioritized foraging problem.
\end{itemize}

%%%%%%%%%%%%%%%%%%%%%%%%%%%%%%%%%%%%%%%%%%%%%%%%%
%%%%%%%%%%%%%%%%%%%%%%%%%%%%%%%%%%%%%%%%%%%%%%%%%


\section{Summary of Conclusions}
\label{sec:conclusions:conclusion_summary}

The first contribution of this research was the introduction of a novel variation of the swarm robotics multi-foraging problem, denoted as the prioritized foraging problem. The prioritized foraging problem differs from other multi-foraging swarm robotics problems in that their exist two types of items which have different priorities. The items with higher priorities should be foraged as fast as possible, while the non-prioritized items should only be foraged to enable the prioritized items to be foraged at a faster rate. The prioritized foraging problem is a novel way of modelling the real-world problem of search and rescue.  

The existing na\"ive and desert ant algorithms were reviewed and selected for evaluation on the prioritized foraging problem. Furthermore, a novel honey bee inspired foraging algorithm was developed, which specifically modelled the recruitment mechanism and division of labour mechanism of honey bee swarms.

The proposed performance measures for the prioritized foraging problem evaluated each algorithm . Furthermore, additional measures were defined so that behavioural analysis of the algorithms could be performed. 

Emperical experiments were defined in Chapter~\ref{chap:experiment}, and an emperical analysis was evaluated in Chapter~\ref{chap:results}. The purpose of the emperical analysis was to evaluate each algorithm's performance on the prioritized foraging problem in terms of the major swarm robotics characteristics of efficiency, scalability, flexibility, and robustness. 

Each algorithm's efficiency was compared to each other algorithm's efficency using a wins and losses approach. Results showed that the honey bee algorithm was the most efficient over all environments and swarm configurations, while the desert ant was the next most efficient and the na\"ive algorithm was the least efficient.

The flexibility of the algorithms was analysed in terms of the flexibility over different environments prioritized item ratio and the flexibility over different environment distribution types, using macro performance measures that were calculated using results of all experiments. Results indicated that the honey bee algorithm was the most flexible, due to its ability to adapt the swarm's specialization ratio to more suitably forage a given environment. The na\"ive algorithm was more flexible than the desert ant algorithm in terms of environment item distribution, which was attributed to the fact that the na\"ive algorithm only appeared flexible because it performs equally bad across all environment item distributions. The honey bee algorithm was the most flexible over different types of environmental distributions, followed by the desert ant algorithm, and lastly the na\"ve algorithm. The honey bee algorithm was determined to be more flexible, due to it's ability to adapt the specialization ratio to better suit the environment ratio of the accessible environment at any point during the foraging process. The honey bee algorithm's adaptation of specialization ratio allowed the swarm to focus on foraging non-prioritized items when only non-prioritized items were accessible, and then adjust the ratio to focus on foraging prioritized items when more prioritized items became accessible, and vice versa. The desert ant and na\"ive algorithms were shown to be less flexible than the honey bee algorithm, due to their inability to adapt the initial swarm specialization ratio.

The scalability was analyzed in terms of swarm scalability, and problem scalability. Swarm scalability was sub-linear for all algorithms. It was established that the na\"ive algorithm was the most scalable in terms of swarm scalability, followed by the desert ant algorithm and honey bee algorithm which exhibited similar scalability. The desert ant algorithm's poor scalability when compared to the na\"ive algorithm, was attributed the desert ant algorithm's use of site fidelity. The desert ant algorithm's site fidelity increased inter-robot interference, which resulted in decreased efficiency when swarm density was high. The honey bee algorithm was shown to be slightly more scalable than the desert ant algorithm, due to the honey bee algorithm's attempt to regulate the number of active foragers by division of labour. Despite the honey bee's ability to regulate the number of active foragers, the difference in swarm scalability between the desert ant algorithm and the honey bee algorithm was small. The honey bee algorithm's ability to regulate the number of active foragers as swarm density increases was shown to be ineffective. 

The honey bee algorithm was the most scalable in terms of the problem density, followed by the na\"ive algorithm, and lastly the desert ant algorithm. Discussion revealed that the desert ant algorithm performed comparatively worse than the na\"ive algorithm in terms of problem scalability, due to the desert ant algorithm's use of site fidelity to exploit good search areas. The na\"ive algorithm's ability to explore more than the desert ant algorithm resulted in better problem scalability. The honey bee algorithm was the most scalable in terms of problem scalability. The problem scalability of the honey bee algorithm was attributed the honey bee algorithm's ability to adapt the swarm specialization ratio to help clear non-prioritized items. By clearing the non-prioritized items, the honey bee algorithm decreased inter-robot and environmental interference, which in turn increased the foraging efficiency in problems of high complexity.

Lastly, robustness of the algorithms was evaluated in terms of redundancy and decentralized coordination. Evidence showed that the honey bee algorithm was the most redundant, because the swarm is completely homogeneous. The division of labour mechanisms of the honey bee swarm allowed each robot to switch their item specialization. The desert ant algorithm and na\"ive algorithm has heterogeneous robots, which resulted in decreased redundancy. The coordination between robots of the desert ant and na\"ive algorithms was determined to be more decentralized than the coordination of robots in the honey bee algorithm. The honey bee algorithm's coordination mechanisms were shown to be sensitive to faults of specific swarm individuals where invalid information was communicated by faulty individuals to the rest of swarm.



%%%%%%%%%%%%%%%%%%%%%%%%%%%%%%%%%%%%%%%%%%%%%%%%%
%%%%%%%%%%%%%%%%%%%%%%%%%%%%%%%%%%%%%%%%%%%%%%%%%

\section{Future Work}
\label{sec:conclusions:future_work}



%Enumerate the future work that you could foresee developing from the work you have done here. Mention areas you could not focus on, or possible extensions to your work. It is a good idea to be thorough, since you increase your chances of being referenced by other researchers who follow up on your work, even if you do not do so yourself. You may consider writing this as a bulleted list, if you mention many aspects.

%%%%%%%%%%%%%%%%%%%%%%%%%%%%%%%%%%%%%%%%%%%%%%%%%
%%%%%%%%%%%%%%%%%%%%%%%%%%%%%%%%%%%%%%%%%%%%%%%%%