%%%%%%%%%%%%%%%%%%%%%%%%%%%%%%%%%%%%%%%%%%%%%%%%%
%%%%%%%%%%%%%%%%%%%%%%%%%%%%%%%%%%%%%%%%%%%%%%%%%

\chapter{Conclusions}
\label{chap:conclusions}

This chapter provides an overview of the conclusions arrived at in this work. For ease of reference, the research objectives which were given at the outset of this thesis are reiterated in Section~\ref{sec:introduction:objectives}. A synopsis of the contributions and experimental findings of this thesis are presented in Section~\ref{sec:conclusions:conclusion_summary}, while Section~\ref{sec:conclusions:future_work} outlines possible future work that can be performed.


\section{Objectives}
\label{sec:introduction:objectives}

The primary objectives of this thesis are as follows: To

\begin{itemize}
	\item conduct a survey of the swarm robotics field;
	\item conduct a survey of foraging in social insects and  swarm robotics;
	\item define the prioritized foraging problem;
	\item propose metrics for evaluating the performance of swarm robotics algorithms on the prioritized foraging problem;
	\item propose and develop different nature-inspired algorithms in a simulated swarm robotic environment, to be evaluated on the prioritized foraging problem; and
	\item evaluate the efficiency, flexibility, scalability, and robustness of the nature-inspired algorithms over different environments and different swarm configurations, on the prioritized foraging problem.
\end{itemize}

%%%%%%%%%%%%%%%%%%%%%%%%%%%%%%%%%%%%%%%%%%%%%%%%%
%%%%%%%%%%%%%%%%%%%%%%%%%%%%%%%%%%%%%%%%%%%%%%%%%


\section{Summary of Conclusions}
\label{sec:conclusions:conclusion_summary}

The first contribution of this research was the introduction of a novel variation of the swarm robotics multi-foraging problem, denoted as the prioritized foraging problem. The prioritized foraging problem differs from other multi-foraging swarm robotics problems in that their exist two types of items which have different priorities. The items with higher priorities should be foraged as fast as possible, while the non-prioritized items should only be foraged to enable the prioritized items to be foraged at a faster rate. The prioritized foraging problem is a novel way of modelling the real-world problem of search and rescue.  

The existing na\"ive and desert ant algorithms were reviewed and selected for evaluation on the prioritized foraging problem. Furthermore, a novel honey bee inspired foraging algorithm was developed, which specifically modelled the recruitment mechanism and division of labour mechanism of honey bee swarms.

Emperical experiments were defined in Chapter~\ref{chap:experiment}, and an emperical analysis was performed in Chapter~\ref{chap:results}. The purpose of the emperical analysis was to investigate each algorithm's performance on the prioritized foraging problem in terms of the major swarm robotics characteristics of efficiency, scalability, flexibility, and robustness as well as the behaviours that enabled those characteristics. 

Each algorithm's efficiency was compared to each other algorithm's efficency using a wins and losses approach. Results showed that the honey bee algorithm was the most efficient over all environments and swarm configurations, while the desert ant was the next most efficient and the na\"ive algorithm was the least efficient.

The flexibility of the algorithms was analysed in terms of the flexibility over environmental item ratio, and the flexibility over different environment distribution types. Results indicated that the honey bee algorithm could adapt the swarm's specialization ratio to effectively forage a given environment item ratio, and was therefore the most flexible in terms of environmental item ratio. The na\"ive algorithm only appeared more flexible than the desert ant algorithm because the nai\"ive performed equally poorly across all environment item distribution, where as the desert algorithm favoured particular item distributions.

The honey bee algorithm was the most flexible over different types of environmental distributions, followed by the desert ant algorithm, and lastly the na\"ve algorithm. The honey bee algorithm's high flexibility was attributed to its adaptation of the specialization ratio to better suit the environment ratio of the accessible environment. The honey bee algorithm's adaptation of specialization ratio allowed the swarm to focus on foraging non-prioritized items when only non-prioritized items were accessible, and then adjust the ratio to focus on foraging prioritized items when more prioritized items became accessible, and vice versa. The desert ant and na\"ive algorithms were less flexible than the honey bee algorithm, due to their inability to adapt the initial swarm specialization ratio.

The scalability of each algorithm was analyzed in terms of swarm scalability, and problem scalability. Swarm scalability was sub-linear for all algorithms, however the na\"ive algorithm was the most scalable. The desert ant algorithm had poor swarm scalability when compared to the na\"ive algorithm. The desert ant algorithm's site fidelity increased inter-robot interference, which resulted in decreased efficiency when swarm density was high. The honey bee algorithm was shown to be slightly more scalable than the desert ant algorithm, due to the honey bee algorithm's attempt to regulate the number of active foragers by division of labour. Future investigation is necessary to determine why the honey bee algorithm did not regulate the number of foragers at swarm high densities very well. 

The honey bee algorithm was the most scalable in terms of the problem density, followed by the na\"ive algorithm, and lastly the desert ant algorithm. The desert ant algorithm performed comparatively better than the na\"ive algorithm in terms of problem scalability. It was theorized that the desert ant algorithm could potentially exploit hard to reach sites in a complex environments. The problem scalability of the honey bee algorithm was attributed the honey bee algorithm's ability to adapt the swarm specialization ratio to help clear non-prioritized items. By clearing the non-prioritized items, the honey bee algorithm decreased inter-robot and environmental interference, which in turn increased the foraging efficiency in problems of high complexity.

Lastly, robustness of each algorithm was evaluated in terms of redundancy and decentralized coordination. A rational argument  motivated that the honey bee algorithm was the most redundant, because the honey bee algorithm swarm is completely homogeneous. The robots of the desert ant algorithm and na\"ive algorithm are heterogeneous, and thus the swarm is less redudant. The coordination between robots of the desert ant and na\"ive algorithms was determined to be more decentralized than the coordination of robots in the honey bee algorithm. Evidence showed that invalid information was communicated by  individuals with faulty information to the rest of swarm, impacting efficiency and showing that communication is less decentralized.

In summation, this research primarily contributed the prioritized foraging problem as well as a novel honey bee foraging algorithm. The emperical analysis provided a detailed view of how well each algorithm embodied the major characteristics of swarm robotics on the prioritized foraging problem.

%%%%%%%%%%%%%%%%%%%%%%%%%%%%%%%%%%%%%%%%%%%%%%%%%
%%%%%%%%%%%%%%%%%%%%%%%%%%%%%%%%%%%%%%%%%%%%%%%%%

\section{Future Work}
\label{sec:conclusions:future_work}

The following is a list of potential future studies that could follow from the work presented in this dissertation:

\begin{enumerate}
    \item A thorough emperical study of the robustness of the described algorithm can be performed. A portion of the swarm can be destroyed during foraging and the impact on the swarm efficiency can be analysed.
    \item An analysis of the impact of using different division of labour strategies for the honey bee algorithm can be done. The study should additionally determine why the division of labour strategy used in this thesis did not regulate the number of active foragers very well.
    \item A more detailed investigation into the performance of each algorithm on each type of environment distribution can be done, to provide more insight into each algorithm's behaviour on different environment types.
    \item An extensive analysis of efficiency of each algorithm, with optimal swarm parameters for each algorithm on each environment can be performed.
    \item The implementation of the described problem and algorithms on a real robot swarm can be done, to determine feasability of the algorithm's in real world application.   
\end{enumerate}

%Enumerate the future work that you could foresee developing from the work you have done here. Mention areas you could not focus on, or possible extensions to your work. It is a good idea to be thorough, since you increase your chances of being referenced by other researchers who follow up on your work, even if you do not do so yourself. You may consider writing this as a bulleted list, if you mention many aspects.

%%%%%%%%%%%%%%%%%%%%%%%%%%%%%%%%%%%%%%%%%%%%%%%%%
%%%%%%%%%%%%%%%%%%%%%%%%%%%%%%%%%%%%%%%%%%%%%%%%%