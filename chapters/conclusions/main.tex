%%%%%%%%%%%%%%%%%%%%%%%%%%%%%%%%%%%%%%%%%%%%%%%%%
%%%%%%%%%%%%%%%%%%%%%%%%%%%%%%%%%%%%%%%%%%%%%%%%%

\chapter{Conclusions}
\label{chap:conclusions}

This chapter provides an overview of the conclusions arrived at in this work. For ease of reference, the research objectives which were given at the outset of this thesis are reiterated in Section~\ref{sec:introduction:objectives}, while Section~\ref{sec:introduction:contributions} summarizes the novel contributions made by the research. A synopsis of the experimental findings of this thesis are presented in Section~\ref{sec:conclusions:conclusion_summary}, while possible future work is  summarises the conclusions derived at by the research performed, while Section~\ref{sec:conclusions:sec:conclusions:future_work} outlines possible future work that can be performed.


\section{Objectives}
\label{sec:introduction:objectives}

The primary objectives of this thesis are as follows: To

\begin{itemize}
	\item conduct a survey of the swarm robotics field;
	\item conduct a survey of foraging in social insects and  swarm robotics;
	\item define the prioritized foraging problem;
	\item propose metrics for evaluating the performance of swarm robotics algorithms on the prioritized foraging problem;
	\item propose and develop different nature-inspired algorithms in a simulated swarm robotic environment, to be evaluated on the prioritized foraging problem; and
	\item evaluate the efficiency, flexibility, scalability, and robustness of the nature-inspired algorithms over different environments and different swarm configurations, on the prioritized foraging problem.
\end{itemize}

%%%%%%%%%%%%%%%%%%%%%%%%%%%%%%%%%%%%%%%%%%%%%%%%%
%%%%%%%%%%%%%%%%%%%%%%%%%%%%%%%%%%%%%%%%%%%%%%%%%


\section{Summary of Conclusions}
\label{sec:conclusions:conclusion_summary}

The first contribution of this research was to introduce a novel variation of the swarm robotics multiforaging problem, denoted the prioritized foraging problem. The prioritized foraging problem differs from other multi-foraging swarm robotics problems in that the two types of items that need to be foraged have different priorities. One type of item has a high priority to be foraged, while the other type of item has a lower priority of being foraged. The prioritized foraging problem is a novel way of modelling the real-world problem of search and rescue.  

This research then reviewed and selected appropriate existing swarm robotics foraging algorithms to be evaluated on the prioritized foraging problem. A novel honeybee inspired foraging algorithm is also developed. The division of labour between of honeybees between different types of food during times of stressed was specifically modelled in the honeybee inspired algorithm.

In order to evaluate the selected algorithms on the prioritized foraging problem, methods for generating prioritized foraging environments, with different environment item type ratios and with clustered, uniform, gaussian, and vein distributions of items, was presented. Performance measures were defined to evaluate each algorithm on the prioritized foraging problem, in terms of their efficiency, scalability, flexibility, and robustness. 

Results showed that the honey bee algorithm was the most efficient over all environments and swarm configurations, while the desert ant was the next most efficient and the na\"ive algorithm was the least efficient.

Flexibility was analysed in terms of flexibility over different environments prioritized item ratio and flexibility over different environment distribution types. The honey bee algorithm was determined to be the most flexible in terms of an environments prioritized item ratio, which was demonstrated to be a result of its ability to adapt the swarm's specialization ratio to more suitably forage a given environment. The na\"ive algorithm was more flexible than the desert ant algorithm in terms of environment item distribution. The na\"ive algorithm's flexibility is attributed to the fact that the na\"ive algorithm is less efficient across all environment item ratios. This suggests that the na\"ive algorithm appears flexible only because it performs equally bad across all environment item distributions.

The honey bee algorithm is the most flexible over different types of environmental distributions, followed by the desert ant algorithm, and lastly the na\"ve algorithm. The honey bee algorithm was determined to be more flexible, due to it's ability to adapt the specialization ratio to better suit the environment ratio of the accessible environment. The adaptation of specialization ratio allowed the honey bee swarm to focus on foraging non-prioritized items when only non-prioritized items were accessible, and then adjust the ratio to focus on foraging prioritized items when more prioritized items became accessible. The desert ant and na\"ive algorithms were shown to be less flexible than the honey bee algorithm. 

The scalability in terms of swarm scalability, was determined to be sub-linear. The na\"ive algorithm was the most scalable in terms of swarm density, followed by the desert ant algorithm and honey bee algorithm which exhibited similar scalability. The desert ant algorithm's poor performance, compared to the na\"ive algorithm, was attributed the desert ant algorithm's use of site fidelity. The desert ant algorithm's site fidelity increased inter-robot interference, which resulted in decreased efficiency when swarm density was high. The honey bee algorithm was shown to be slightly more scalable than the desert ant algorithm, due to the honey bee algorithm's attempt to regulate the number of active foragers by division of labour. The difference in swarm scalability between the desert ant algorithm and the honey bee algorithm was very small, suggesting that the regulation of active foragers was not functioning very well. The honey bee algorithm's ability to regulate the number of active foragers as swarm density increases was shown to be ineffective. 

The honey bee algorithm was the most scalable in terms of the problem density , followed by the na\"ive algorithm, and lastly the desert ant algorithm. Discussion revealed that the desert ant algorithm performed comparatively worse than the na\"ive algorithm in terms of problem scalability, due to the desert ant algorithm's use of site fidelity to exploit good search areas. The section determined the na\"ive algorithm's ability to explore more than the desert ant algorithm resulted in better problem scalability. The honey bee algorithm was the most scalable in terms of problem scalability. The problem scalability of the honey bee algorithm can be attributed to the ability of the honey bee algorithm to adapt $\tau$ to help clear non-prioritized items, which resulted in lower inter-robot and environmental interference, which in turn increased the foraging efficiency.

Robustness of the algorithms was evaluated in terms of redundancy and decentralized coordination. Evidence showed that the honey bee algorithm was the most redundant, because the swarm is completely homogeneous. The division of labour mechanisms of the honey bee swarm allow each robot to switch their item specialization. The desert ant algorithm and na\"ive algorithm robots can only forage a single item type, which is pre-configured, and thus their swarms are heterogeneous, resulting in lower redundancy.

The coordination between robots of the desert ant and na\"ive algorithms was determined to be more decentralized than the coordination of robots in the honey bee algorithm. The honey bee algorithm's coordination mechanisms were shown to be sensitive to faults in specific swarm individuals where invalid information was communicated to the swarm, and the communication of invalid information impacted the overall swarm efficiency. 


%%%%%%%%%%%%%%%%%%%%%%%%%%%%%%%%%%%%%%%%%%%%%%%%%
%%%%%%%%%%%%%%%%%%%%%%%%%%%%%%%%%%%%%%%%%%%%%%%%%

\section{Future Work}
\label{sec:conclusions:future_work}

%Enumerate the future work that you could foresee developing from the work you have done here. Mention areas you could not focus on, or possible extensions to your work. It is a good idea to be thorough, since you increase your chances of being referenced by other researchers who follow up on your work, even if you do not do so yourself. You may consider writing this as a bulleted list, if you mention many aspects.

%%%%%%%%%%%%%%%%%%%%%%%%%%%%%%%%%%%%%%%%%%%%%%%%%
%%%%%%%%%%%%%%%%%%%%%%%%%%%%%%%%%%%%%%%%%%%%%%%%%