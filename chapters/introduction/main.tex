%%%%%%%%%%%%%%%%%%%%%%%%%%%%%%%%%%%%%%%%%%%%%%%%%
%%%%%%%%%%%%%%%%%%%%%%%%%%%%%%%%%%%%%%%%%%%%%%%%%

\chapter{Introduction}
\label{chap:introduction}
\pagestyle{headings}
\pagenumbering{arabic}
\setcounter{page}{1}

%%%%%%%%%%%%%%%%%%%%%%%%%%%%%%%%%%%%%%%%%%%%%%%%%
%%%%%%%%%%%%%%%%%%%%%%%%%%%%%%%%%%%%%%%%%%%%%%%%%

Section~\ref{sec:introduction:motivation} motivates the purpose of this study, while objectives of this research are highlighted in Section~\ref{sec:introduction:objectives}. The contribution of the thesis are presented in Section~\ref{sec:introduction:contributions}, while Section~\ref{sec:introduction:outline} outlines the the structure of the thesis.
%%%%%%%%%%%%%%%%%%%%%%%%%%%%%%%%%%%%%%%%%%%%%%%%%
%%%%%%%%%%%%%%%%%%%%%%%%%%%%%%%%%%%%%%%%%%%%%%%%%

\section{Motivation}
\label{sec:introduction:motivation}

Swarm robotics is the co-coordination of large numbers of  relatively simple robots to perform a single collaborative function. Swarm robotics is inspired from the observation of social insects such as ants, termites, and bees \cite{dorigo2004swarm}. An important activity of all natural swarms is foraging for resources. Foraging is defined as the search and collection of resources from sources in an environment and returning the resources to a collection point \cite{winfield2009foraging}. These resources could be food, water, or building materials. %TODO: Citation
% and is formally defined in \cite{stephens1986foraging}. 

In times of stress, the collection of one resource may be prioritized over others - such as water during a drought or food before winter. Individuals in a natural swarm often adapt behaviour appropriately to enable greater collection of the prioritized item.
%TODO: Citations? Or provide later

Item prioritization during foraging exists in real-world robot foraging problems such as search and rescue and gold mining. In the case of a building collapsing, robots need to get to the survivors as quickly as possible; however, it is important that some robots move waste material quickly in order to reach the trapped survivors. Prioritized foraging is also similar to gold mining, where gold needs to be foraged as a priority, and the waste needs to be moved out of the way. In the mentioned applications, the distribution of objects and ratios of prioritized to non-prioritized types of items is usually unknown. Seeing a search and rescue situation or mining problem as a foraging problem, which contains items of differing priority 

In order to address this, this study selects and develops different nature-based swarm robotics algorithms, and evaluated the algorithms' performance on the prioritized foraging problem. The properties of each algorithm is then addressed in order to further the understanding of the prioritized item foraging problem and identify robot behaviours that are efficient, scalable, flexible and robust.

%%%%%%%%%%%%%%%%%%%%%%%%%%%%%%%%%%%%%%%%%%%%%%%%%
%%%%%%%%%%%%%%%%%%%%%%%%%%%%%%%%%%%%%%%%%%%%%%%%%

\section{Objectives}
\label{sec:introduction:objectives}

The primary objectives of this thesis are as follows:

\begin{itemize}
	\item Conduct a survey of the swarm robotics field
	\item Conduct a survey of foraging in social insects and  swarm robotics.
	\item Define the prioritized foraging problem.
	\item Selection and implementation of different nature-inspired algorithms in a simulated swarm robotic environment, to be evaluated on the prioritized foraging problem.
	\item Evaluate and contrast the efficiency, flexibility, scalability, and robustness of the nature-inspired algorithms over different environments and different swarm configurations.
\end{itemize}


%%%%%%%%%%%%%%%%%%%%%%%%%%%%%%%%%%%%%%%%%%%%%%%%%
%%%%%%%%%%%%%%%%%%%%%%%%%%%%%%%%%%%%%%%%%%%%%%%%%

\section{Contributions}
\label{sec:introduction:contributions}

\begin{itemize}
	\item A novel variation of the foraging problem - prioritized foraging.
	\item A novel desert ant prioritized foraging algorithm for robot swarms.
	\item A novel honey bee inspired prioritized foraging algorithm for robot swarms.
	\item Methods for generating various types of foraging environments.
	\item An analysis of the efficiency, flexibility, scalability, and robustness of each algorithm  
\end{itemize}


%%%%%%%%%%%%%%%%%%%%%%%%%%%%%%%%%%%%%%%%%%%%%%%%%
%%%%%%%%%%%%%%%%%%%%%%%%%%%%%%%%%%%%%%%%%%%%%%%%%

\section{Thesis Outline}
\label{sec:introduction:outline}
This thesis consists of 8 chapters, each handling a separate topic. The chapters are as follows:

\begin{itemize}
\item\textbf{Chapter~\ref{chap:first}} introduces the user to swarm robotics in terms of its origin, motivation, development and current state.

\item\textbf{Chapter~\ref{chap:second}} defines foraging. The chapter summarizes foraging behaviour of social insects and reviews existing foraging algorithms in swarm robotics.

\item\textbf{Chapter~\ref{chap:divisionoflabour}} defines division of labour and summarizes strategies for division of labour employed by social insects. The chapter reviews where division of labour strategies have been employed by robot swarms.

\item\textbf{Chapter~\ref{chap:third}} introduces a novel foraging variation called prioritized foraging. The chapter presents three swarm robotic foraging algorithms inspired by social insects, whose performance is to be evaluated on prioritized foraging problem.

\item\textbf{Chapter~\ref{chap:experiment}} presents an experiment designed to evaluate the discussed foraging algorithms in a simplified simulation environment on the prioritized foraging problem. This chapter describes the environmental parameters, algorithm parameters and the swarm parameters that the algorithms are run with. This chapter presents the relevant performance measures to be used to evaluate the discussed algorithms. This chapter also describes the capabilities of the simulated robots as well .

\item\textbf{Chapter~\ref{chap:results}} discusses and summarizes the results of applying the discussed foraging algorithms to the prioritized foraging problem

\item\textbf{Chapter~\ref{chap:conclusions}} summarizes the findings of this research.

\end{itemize}

The thesis contains 3 appendices, which are organised as follows:

\begin{itemize}
\item\textbf{Appendix~\ref{app:acronyms}} provides a list of important acronyms that are used in the course of this work, along with their definitions

\item\textbf{Appendix~\ref{app:symbols}} lists and defines the mathematical symbols used in this thesis, divided according to the chapter in which they appear.

\item\textbf{Appendix~\ref{app:derived_publications}} contains the publications derived from this work.

\end{itemize}




%You might also include a page reference to the index (if you decide to include one) here, as follows: page~\pageref{index}.

%Please refer to the bibliography database (in the file \texttt{bibliography.bib}) for a skeleton of how you would add references to your work. You cite them in your text, in any order you need, as follows: \cite{ref:Alhoniemi:1999b}, \cite{ref:Alahakoon:2000}, \cite{ref:Oja:2003}, \cite{ref:Alhoniemi:1999a}, \cite{ref:Kaski:1995}, \cite{ref:Fasulo:1999}, \cite{ref:Kaski:1997} and \cite{ref:Aha:1998}. You may include extra page information as follows: \cite{ref:Clark:1989}[page~82]. Note that references will only show up in the bibliography once you actually cite them in the document. The entries will also be alphabetised automatically.

%Make sure that the bibliographic information is as accurate as possible. Provide full author names, as they appear on the paper. Make sure you provide as much information as possible, but try to keep it relatively brief as well (for example, include minimal information in the address field). Make absolutely sure that references are correct, since there are a large number of incorrect ones listed in published articles.

%Note that references to online resources should only be provided in instances when a document's primary publication method is online. You may also provide a reference to a Digital Object Identifier (DOI)\footnote{A DOI is a unique alphanumeric identification string for a digital object, providing a persistent link to it. It represents a permanent URL kept the same way a domain name is. For further information on DOI, see \url{http://www.doi.org}. Publication DOIs are provided in the CrossRef framework. CrossRef is a non-profit network providing infrastructure for linking online citations, using the DOIs of documents that are available electronically. CrossRef DOIs looks something like \texttt{doi:10.1234/5678}, and work like standard hyperlinks in most web browsers (you can past them into the address bar of a browser and you will be taken directly to the online document). DOIs may also be manually resolved via \url{http://dx.doi.org}. For more information on CrossRef, see \url{http://www.crossref.org}.} if one is available, but this should always be provided in conjunction with the physical publication details (note that many journals and conferences will still have a problem with DOIs being cited, since they are online documents, even though they are persistent).

%After you have discussed the chapters, provide a brief introduction for the list of appendices:
%\begin{itemize}
%	\item\textbf{Appendix~\ref{app:appendix1}} describes \ldots.
	
%	\item\textbf{Appendix~\ref{app:appendix2}} covers \ldots.
	
%	\item\textbf{Appendix~\ref{app:acronyms}} provides a list of the important acronyms used or newly defined in the course of this work, as well as their associated definitions.
	
%	\item\textbf{Appendix~\ref{app:symbols}} lists and defines the mathematical symbols used in this work, categorised according to the relevant chapter in which they appear.
	
%	\item\textbf{Appendix~\ref{app:derived_publications}} lists the publications derived from this work.
%\end{itemize}

%If you provide an index, you may provide a page reference here, indicating that it begins on page~\pageref{index} of the text.

%%%%%%%%%%%%%%%%%%%%%%%%%%%%%%%%%%%%%%%%%%%%%%%%%
%%%%%%%%%%%%%%%%%%%%%%%%%%%%%%%%%%%%%%%%%%%%%%%%%