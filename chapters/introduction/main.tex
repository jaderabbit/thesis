%%%%%%%%%%%%%%%%%%%%%%%%%%%%%%%%%%%%%%%%%%%%%%%%%
%%%%%%%%%%%%%%%%%%%%%%%%%%%%%%%%%%%%%%%%%%%%%%%%%

\chapter{Introduction}
\label{chap:introduction}
\pagestyle{headings}
\pagenumbering{arabic}
\setcounter{page}{1}

%%%%%%%%%%%%%%%%%%%%%%%%%%%%%%%%%%%%%%%%%%%%%%%%%
%%%%%%%%%%%%%%%%%%%%%%%%%%%%%%%%%%%%%%%%%%%%%%%%%

\begin{quote}
	{\it You may place a brief sketch of the scenario that inspired the research here (you may also choose to leave it out). For example, if you are writing about ACOs, you might write something about the simplicity and efficiency of an ant colony. This may be relatively informal (but not colloquial). Avoid references here. Refer to some of the theses on the CIRG website for some ideas on what you might include.}
\end{quote}
%Place some very general background information here, setting the scene for where your work fits in tot he broader scheme of things.

%For instance, give a very broad overview of the field of CI-based function optimisation. You should already provide some references (here's an example of a reference \cite{ref:Engelbrecht:2002}).

%%%%%%%%%%%%%%%%%%%%%%%%%%%%%%%%%%%%%%%%%%%%%%%%%
%%%%%%%%%%%%%%%%%%%%%%%%%%%%%%%%%%%%%%%%%%%%%%%%%

\section{Motivation}
\label{sec:introduction:motivation}

Swarm robotics is the co-coordination of large numbers of  relatively simple robots to perform a single collaborative function, inspired from the observation of social insects such as ants, termites, and bees \cite{dorigo2004swarm}. An important activity of all natural swarms is foraging for resources. Foraging is defined as the search and collection of resources from sources in an environment and returning the resources to a collection point \cite{winfield2009foraging}. These resources could be food, water, or building materials. %TODO: Citation
% and is formally defined in \cite{stephens1986foraging}. 

In times of stress, the collection of one resource may be prioritized over others - such as water during a drought or food before winter. Individuals in a natural swarm often adapt behaviour appropriately to enable greater collection of the prioritized item.
%TODO: Citations? Or provide later

Item prioritization during foraging exists in real-world robot foraging problems such as search and rescue and gold mining. In the case of a building collapsing, robots need to get to the survivors as quickly as possible; however, it is important that some robots move waste material to reach the trapped survivors. Prioritized foraging can also be applied to the gold mining problem where  gold needs to be foraged as a priority and the waste needs to be moved out of the way. In the mentioned applications, the distribution of objects and ratios of prioritized to non-prioritized types is often unknown.

%%%%%%%%%%%%%%%%%%%%%%%%%%%%%%%%%%%%%%%%%%%%%%%%%
%%%%%%%%%%%%%%%%%%%%%%%%%%%%%%%%%%%%%%%%%%%%%%%%%

\section{Objectives}
\label{sec:introduction:objectives}

Introduce the research objectives of your work. It is quite likely that these may only become clear after you have completed the rest of the thesis, so revisit this list once you have finished everything else. Include broad objectives, like ``conduct a survey of available techniques in the field of \ldots'', or ``present practical case studies in the realm of \ldots''. You should use a bulleted list, as follows:
\begin{itemize}
	\item First objective here.

	\item Then, second objective here.

	\item Third objective here, and so on.
\end{itemize}

%%%%%%%%%%%%%%%%%%%%%%%%%%%%%%%%%%%%%%%%%%%%%%%%%
%%%%%%%%%%%%%%%%%%%%%%%%%%%%%%%%%%%%%%%%%%%%%%%%%

\section{Contributions}
\label{sec:introduction:contributions}

%Enumerate the novel contributions that your work sets out to make to the field. Include specific novel contributions to the field, such as taxonomies, or empirical results. These will be quite closely related to the objectives listed in Section~\ref{sec:introduction:objectives}. You may also use a bulleted list here.

%%%%%%%%%%%%%%%%%%%%%%%%%%%%%%%%%%%%%%%%%%%%%%%%%
%%%%%%%%%%%%%%%%%%%%%%%%%%%%%%%%%%%%%%%%%%%%%%%%%

\section{Thesis Outline}
\label{sec:introduction:outline}

%Introduce a list of the remaining chapters of your work, in which a brief (two to three line) description of the material covered in each is included:

%You might also include a page reference to the index (if you decide to include one) here, as follows: page~\pageref{index}.

%Please refer to the bibliography database (in the file \texttt{bibliography.bib}) for a skeleton of how you would add references to your work. You cite them in your text, in any order you need, as follows: \cite{ref:Alhoniemi:1999b}, \cite{ref:Alahakoon:2000}, \cite{ref:Oja:2003}, \cite{ref:Alhoniemi:1999a}, \cite{ref:Kaski:1995}, \cite{ref:Fasulo:1999}, \cite{ref:Kaski:1997} and \cite{ref:Aha:1998}. You may include extra page information as follows: \cite{ref:Clark:1989}[page~82]. Note that references will only show up in the bibliography once you actually cite them in the document. The entries will also be alphabetised automatically.

%Make sure that the bibliographic information is as accurate as possible. Provide full author names, as they appear on the paper. Make sure you provide as much information as possible, but try to keep it relatively brief as well (for example, include minimal information in the address field). Make absolutely sure that references are correct, since there are a large number of incorrect ones listed in published articles.

%Note that references to online resources should only be provided in instances when a document's primary publication method is online. You may also provide a reference to a Digital Object Identifier (DOI)\footnote{A DOI is a unique alphanumeric identification string for a digital object, providing a persistent link to it. It represents a permanent URL kept the same way a domain name is. For further information on DOI, see \url{http://www.doi.org}. Publication DOIs are provided in the CrossRef framework. CrossRef is a non-profit network providing infrastructure for linking online citations, using the DOIs of documents that are available electronically. CrossRef DOIs looks something like \texttt{doi:10.1234/5678}, and work like standard hyperlinks in most web browsers (you can past them into the address bar of a browser and you will be taken directly to the online document). DOIs may also be manually resolved via \url{http://dx.doi.org}. For more information on CrossRef, see \url{http://www.crossref.org}.} if one is available, but this should always be provided in conjunction with the physical publication details (note that many journals and conferences will still have a problem with DOIs being cited, since they are online documents, even though they are persistent).

%After you have discussed the chapters, provide a brief introduction for the list of appendices:
%\begin{itemize}
%	\item\textbf{Appendix~\ref{app:appendix1}} describes \ldots.
	
%	\item\textbf{Appendix~\ref{app:appendix2}} covers \ldots.
	
%	\item\textbf{Appendix~\ref{app:acronyms}} provides a list of the important acronyms used or newly defined in the course of this work, as well as their associated definitions.
	
%	\item\textbf{Appendix~\ref{app:symbols}} lists and defines the mathematical symbols used in this work, categorised according to the relevant chapter in which they appear.
	
%	\item\textbf{Appendix~\ref{app:derived_publications}} lists the publications derived from this work.
%\end{itemize}

%If you provide an index, you may provide a page reference here, indicating that it begins on page~\pageref{index} of the text.

%%%%%%%%%%%%%%%%%%%%%%%%%%%%%%%%%%%%%%%%%%%%%%%%%
%%%%%%%%%%%%%%%%%%%%%%%%%%%%%%%%%%%%%%%%%%%%%%%%%