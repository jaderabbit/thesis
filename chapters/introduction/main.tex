%%%%%%%%%%%%%%%%%%%%%%%%%%%%%%%%%%%%%%%%%%%%%%%%%
%%%%%%%%%%%%%%%%%%%%%%%%%%%%%%%%%%%%%%%%%%%%%%%%%

\chapter{Introduction}
\label{chap:introduction}
\pagestyle{headings}
\pagenumbering{arabic}
\setcounter{page}{1}

%%%%%%%%%%%%%%%%%%%%%%%%%%%%%%%%%%%%%%%%%%%%%%%%%
%%%%%%%%%%%%%%%%%%%%%%%%%%%%%%%%%%%%%%%%%%%%%%%%%

Section~\ref{sec:introduction:motivation} motivates the purpose of this thesis, and the objectives of this thesis are highlighted in Section~\ref{sec:introduction:objectives}. The contributions of the thesis are presented in Section~\ref{sec:introduction:contributions}, while Section~\ref{sec:introduction:outline} outlines the the structure of the thesis.
%%%%%%%%%%%%%%%%%%%%%%%%%%%%%%%%%%%%%%%%%%%%%%%%%
%%%%%%%%%%%%%%%%%%%%%%%%%%%%%%%%%%%%%%%%%%%%%%%%%

\section{Motivation}
\label{sec:introduction:motivation}
Consider a search and rescue mission after a natural disaster or mining accident. These search and rescue missions are usually extremely dangerous for the human rescuers who are deployed to search for survivors. The use of swarm robotics to perform such search and rescue mission has been proposed and explored as an alternative to using human rescuers \cite{murphy2008search,naghsh2008analysis}.

Swarm robotics is the co-coordination of large numbers of  relatively simple robots to perform a single collaborative function. Swarm robotics is inspired from the observation of social insects such as ants, termites, and bees \cite{dorigo2004swarm}. An important activity of all natural swarms is foraging for resources. Foraging is defined as the search and collection of resources from sources in an environment and returning the resources to a collection point \cite{winfield2009foraging}. These resources could be food, water, or building materials. Foraging is an abstraction of the search and rescue problem where the trapped humans are items that robots need to located and remove to a safe location. 

In a search and rescue mission, the location of trapped humans is usually unknown and humans can be potentially blocked by rubble, which needs to be removed before the humans can be safely removed. A swarm of robots would need to search for the humans, as fast as possible, to avoid further danger, injury or loss of life. If robots cannot locate the humans, they will have to begin clearing debris in the hope of locating them, as well as clearing the route between the trapped humans and the safe zone. Locating and removing the humans is prioritized above removal of the debris, but often the debris must be removed, in order to locate the trapped humans, thus there exists a resource prioritization from the perspective of the robot.

Resource prioritization is also relevant in mining problems where the metal ore is prioritized and the waste rock should be cleared to better access the valuable ore. In common gold mining techniques, the ore and waste rock are collected and transported to the surface and chemical techniques are used to separate them. The transport of the waste rock (which forms majority of the load) to the surface is extremely expensive, so there is a cost benefit to separate the ore and the waste rock beneath the surface and transport only the valuable ore to the surface.

The above search and rescue problem and mining problem can be abstracted as a foraging problem where there exists resources with differing priorities. In the case of search and rescue, the humans are the prioritized resource and the debris is the non-prioritized resource. In mining, the ore is the prioritized resource and the waste rock is the non-prioritized resource. 

In nature, in times of stress, the collection of one resource may be prioritized over others - such as water during a drought or food before winter. Individuals in a natural swarm often adapt behaviour appropriately to enable greater collection of the prioritized item.

This study defines the prioritized foraging problem, and proposes three swarm robotics foraging algorithms, to be evaluated on the prioritized foraging problem. The study proposes metrics to evaluate each algorithm's performance on the prioritized foraging problem in terms of foraging efficiency, flexibility, scalability and robustness. The algorithms are developed in a simulated environment. Each algorithm's performance is evaluated on environments of a variety of complexities, with various swarm configurations.

%%%%%%%%%%%%%%%%%%%%%%%%%%%%%%%%%%%%%%%%%%%%%%%%%
%%%%%%%%%%%%%%%%%%%%%%%%%%%%%%%%%%%%%%%%%%%%%%%%%

\section{Objectives}
\label{sec:introduction:objectives}

The primary objectives of this thesis are as follows: To

\begin{itemize}
	\item conduct a survey of the swarm robotics field;
	\item conduct a survey of foraging in social insects and  swarm robotics;
	\item define the prioritized foraging problem;
	\item propose metrics for evaluating the performance of swarm robotics algorithms on the prioritized foraging problem;
	\item propose and develop different nature-inspired algorithms in a simulated swarm robotic environment, to be evaluated on the prioritized foraging problem; and
	\item evaluate the efficiency, flexibility, scalability, and robustness of the nature-inspired algorithms over different environments and different swarm configurations, on the prioritized foraging problem.
\end{itemize}


%%%%%%%%%%%%%%%%%%%%%%%%%%%%%%%%%%%%%%%%%%%%%%%%%
%%%%%%%%%%%%%%%%%%%%%%%%%%%%%%%%%%%%%%%%%%%%%%%%%

\section{Contributions}
\label{sec:introduction:contributions}

The thesis makes the following contributions:

\begin{itemize}
	\item A novel variation of the foraging problem i.e. prioritized foraging, as well as performance measures for the prioritized foraging problem are proposed.
	\item A novel desert ant inspired foraging algorithm for robot swarms is developed.
	\item A novel honey bee inspired foraging algorithm for robot swarms is developed.
	\item A novel obstacle avoidance algorithm, inspired by the flocking behaviour of birds is developed.
	\item Methods for generating various types of foraging environments in a simulated environments are proposed.
	\item An analysis of the efficiency, flexibility, scalability, and robustness of each algorithm on the prioritized foraging problem is conducted.
\end{itemize}


%%%%%%%%%%%%%%%%%%%%%%%%%%%%%%%%%%%%%%%%%%%%%%%%%
%%%%%%%%%%%%%%%%%%%%%%%%%%%%%%%%%%%%%%%%%%%%%%%%%

\section{Thesis Outline}
\label{sec:introduction:outline}
This thesis consists of eight chapters, each handling a separate topic. The chapters are as follows:

\begin{itemize}
\item\textbf{Chapter~\ref{chap:first}} introduces the origin, motivation, development and current state of swarm robotics.

\item\textbf{Chapter~\ref{chap:second}} summarizes foraging behaviour of social insects and reviews existing foraging algorithms in swarm robotics.

\item\textbf{Chapter~\ref{chap:divisionoflabour}} defines division of labour and summarizes strategies for division of labour employed by social insects. The chapter reviews where division of labour strategies have been employed by robot swarms.

\item\textbf{Chapter~\ref{chap:third}} proposes a novel foraging variation called prioritized foraging. The chapter also proposes three swarm robotic foraging algorithms, inspired by social insects, to be evaluated on a prioritized foraging problem.

\item\textbf{Chapter~\ref{chap:experiment}} presents an experiment designed to evaluate the proposed foraging algorithms in a simplified simulation environment, on the prioritized foraging problem. This chapter describes the environment and swarm parameters that the experiments use. This chapter presents the relevant performance measures to be used to evaluate the proposed algorithms on the prioritized foraging problem.

\item\textbf{Chapter~\ref{chap:results}} analyses the results of applying the proposed algorithms to the prioritized foraging problem

\item\textbf{Chapter~\ref{chap:conclusions}} summarizes the findings of this research.

\end{itemize}

The thesis contains 2 appendices, which are organised as follows:

\begin{itemize}
\item\textbf{Appendix~\ref{app:acronyms}} provides a list of important acronyms that are used in the course of this work, along with their definitions

\item\textbf{Appendix~\ref{app:symbols}} lists and defines the mathematical symbols used in this thesis, divided according to the chapter in which they appear.

%\item\textbf{Appendix~\ref{app:derived_publications}} contains the publications derived from this work.

\end{itemize}




%You might also include a page reference to the index (if you decide to include one) here, as follows: page~\pageref{index}.

%Please refer to the bibliography database (in the file \texttt{bibliography.bib}) for a skeleton of how you would add references to your work. You cite them in your text, in any order you need, as follows: \cite{ref:Alhoniemi:1999b}, \cite{ref:Alahakoon:2000}, \cite{ref:Oja:2003}, \cite{ref:Alhoniemi:1999a}, \cite{ref:Kaski:1995}, \cite{ref:Fasulo:1999}, \cite{ref:Kaski:1997} and \cite{ref:Aha:1998}. You may include extra page information as follows: \cite{ref:Clark:1989}[page~82]. Note that references will only show up in the bibliography once you actually cite them in the document. The entries will also be alphabetised automatically.

%Make sure that the bibliographic information is as accurate as possible. Provide full author names, as they appear on the paper. Make sure you provide as much information as possible, but try to keep it relatively brief as well (for example, include minimal information in the address field). Make absolutely sure that references are correct, since there are a large number of incorrect ones listed in published articles.

%Note that references to online resources should only be provided in instances when a document's primary publication method is online. You may also provide a reference to a Digital Object Identifier (DOI)\footnote{A DOI is a unique alphanumeric identification string for a digital object, providing a persistent link to it. It represents a permanent URL kept the same way a domain name is. For further information on DOI, see \url{http://www.doi.org}. Publication DOIs are provided in the CrossRef framework. CrossRef is a non-profit network providing infrastructure for linking online citations, using the DOIs of documents that are available electronically. CrossRef DOIs looks something like \texttt{doi:10.1234/5678}, and work like standard hyperlinks in most web browsers (you can past them into the address bar of a browser and you will be taken directly to the online document). DOIs may also be manually resolved via \url{http://dx.doi.org}. For more information on CrossRef, see \url{http://www.crossref.org}.} if one is available, but this should always be provided in conjunction with the physical publication details (note that many journals and conferences will still have a problem with DOIs being cited, since they are online documents, even though they are persistent).

%After you have discussed the chapters, provide a brief introduction for the list of appendices:
%\begin{itemize}
%	\item\textbf{Appendix~\ref{app:appendix1}} describes \ldots.
	
%	\item\textbf{Appendix~\ref{app:appendix2}} covers \ldots.
	
%	\item\textbf{Appendix~\ref{app:acronyms}} provides a list of the important acronyms used or newly defined in the course of this work, as well as their associated definitions.
	
%	\item\textbf{Appendix~\ref{app:symbols}} lists and defines the mathematical symbols used in this work, categorised according to the relevant chapter in which they appear.
	
%	\item\textbf{Appendix~\ref{app:derived_publications}} lists the publications derived from this work.
%\end{itemize}

%If you provide an index, you may provide a page reference here, indicating that it begins on page~\pageref{index} of the text.

%%%%%%%%%%%%%%%%%%%%%%%%%%%%%%%%%%%%%%%%%%%%%%%%%
%%%%%%%%%%%%%%%%%%%%%%%%%%%%%%%%%%%%%%%%%%%%%%%%%